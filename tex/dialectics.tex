\chapter{Dialectics}

In the meantime, along with and after the French philosophy of the 18th century,
had arisen the new German philosophy, culminating in Hegel.

Its greatest merit was the taking up again of dialectics as the highest form of
reasoning. The old Greek philosophers were all born natural dialecticians, and
Aristotle, the most encylopedic of them, had already analysed the most essential
forms of dialectic thought. The newer philosophy, on the other hand, although in
it also dialectics had brilliant exponents (e.g. Descartes and Spinoza), had,
especially through English influence, become more and more rigidly fixed in the
so-called metaphysical mode of reasoning, by which also the French of the 18th
century were almost wholly dominated, at all events in their special
philosophical work. Outside philosophy in the restricted sense, the French
nevertheless produced masterpieces of dialectic. We need only call to mind
Diderot's \emph{Le Neveu de Rameau}, and Rousseau's \emph{Discours sur l'origine
et les fondements de l'inegalite parmi less hommes}. We give here, in brief, the
essential character of these two modes of thought.

When we consider and reflect upon Nature at large, or the history of mankind, or
our own intellectual activity, at first we see the picture of an endless
entanglement of relations and reactions, permutations and combinations, in which
nothing remains what, where and as it was, but everything moves, changes, comes
into being and passes away. We see, therefore, at first the picture as a whole,
with its individual parts still more or less kept in the background; we observe
the movements, transitions, connections, rather than the things that move,
combine, and are connected. This primitive, naive but intrinsically correct
conception of the world is that of ancient Greek philosophy, and was first
clearly formulated by Heraclitus: everything is and is not, for everything is
fluid, is constantly changing, constantly coming into being and passing
away\endnote{
  Unknown to the Western world until the 20th century, the Chinese philosopher
  Lao Tzu was a predecessor of or possible contemporary to Heraclitus. Lao Tzu
  wrote the renowned Tao Te Ching in which he also espouses the fundamental
  principles of dialectics.
}.

\printendnotes