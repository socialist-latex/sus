\chapter{Dialectics}

In the meantime, along with and after the French philosophy of the 18th century,
had arisen the new German philosophy, culminating in Hegel.

Its greatest merit was the taking up again of dialectics as the highest form of
reasoning. The old Greek philosophers were all born natural dialecticians, and
Aristotle, the most encylopedic of them, had already analysed the most essential
forms of dialectic thought. The newer philosophy, on the other hand, although in
it also dialectics had brilliant exponents (e.g. Descartes and Spinoza), had,
especially through English influence, become more and more rigidly fixed in the
so-called metaphysical mode of reasoning, by which also the French of the 18th
century were almost wholly dominated, at all events in their special
philosophical work. Outside philosophy in the restricted sense, the French
nevertheless produced masterpieces of dialectic. We need only call to mind
Diderot's \emph{Le Neveu de Rameau}, and Rousseau's \emph{Discours sur l'origine
et les fondements de l'inegalite parmi less hommes}. We give here, in brief, the
essential character of these two modes of thought.

When we consider and reflect upon Nature at large, or the history of mankind, or
our own intellectual activity, at first we see the picture of an endless
entanglement of relations and reactions, permutations and combinations, in which
nothing remains what, where and as it was, but everything moves, changes, comes
into being and passes away. We see, therefore, at first the picture as a whole,
with its individual parts still more or less kept in the background; we observe
the movements, transitions, connections, rather than the things that move,
combine, and are connected. This primitive, naive but intrinsically correct
conception of the world is that of ancient Greek philosophy, and was first
clearly formulated by Heraclitus: everything is and is not, for everything is
fluid, is constantly changing, constantly coming into being and passing
away\endnote{
  Unknown to the Western world until the 20th century, the Chinese philosopher
  Lao Tzu was a predecessor of or possible contemporary to Heraclitus. Lao Tzu
  wrote the renowned Tao Te Ching in which he also espouses the fundamental
  principles of dialectics.
}.

But this conception, correctly as it expresses the general character of the
picture of appearances as a whole, does not suffice to explain the details of
which this picture is made up, and so long as we do not understand these, we
have not a clear idea of the whole picture. In order to understand these
details, we must detach them from their natural, special causes, effects, etc.
This is, primarily, the task of natural science and historical research:
branches of science which the Greek of classical times, on very good grounds,
relegated to a subordinate position, because they had first of all to collect
materials for these sciences to work upon. A certain amount of natural and
historical material must be collected before there can be any critical
analysis, comparison, and arrangement in classes, orders, and species. The
foundations of the exact natural sciences were, therefore, first worked out by
the Greeks of the Alexandrian period\endnote{
  The Alexandrian period of development of science comprises the period
  extending from the 3rd century B.C. to the 17th century A.D. It derives its
  name from the town of Alexandria in Egypt, which was one of the most
  important centres of international economic intercourses at that time. In the
  Alexandrian period, mathematics (Euclid and Archimedes), geography,
  astronomy, anatomy, physiology, etc., attained considerable development.
}, and later on, in the Middle Ages, by the Arabs. Real natural science dates
from the second half of the 15th century, and thence onward it had advanced
with constantly increasing rapidity. The analysis of Nature into its individual
parts, the grouping of the different natural processes and objects in definite
classes, the study of the internal anatomy of organized bodies in their
manifold forms---these were the fundamental conditions of the gigantic strides
in our knowledge of Nature that have been made during the last 400 years. But
this method of work has also left us as legacy the habit observing natural
objects and processes in isolation, apart from their connection with the vast
whole, of observing them in repose, not in motion; as constraints, not as
essentially variables; in their death, not in their life. And when this way of
looking at things was transferred by Bacon and Locke from natural science to
philosophy, it begot the narrow, metaphysical mode of thought peculiar to the
last century.

\printendnotes
