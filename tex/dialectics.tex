\chapter{Dialectics}

In the meantime, along with and after the French philosophy of the 18th century,
had arisen the new German philosophy, culminating in Hegel.

Its greatest merit was the taking up again of dialectics as the highest form of
reasoning. The old Greek philosophers were all born natural dialecticians, and
Aristotle, the most encyclopedic of them, had already analysed the most essential
forms of dialectic thought. The newer philosophy, on the other hand, although in
it also dialectics had brilliant exponents (e.g., Descartes and Spinoza), had,
especially through English influence, become more and more rigidly fixed in the
so-called metaphysical mode of reasoning, by which also the French of the 18th
century were almost wholly dominated, at all events in their special
philosophical work. Outside philosophy in the restricted sense, the French
nevertheless produced masterpieces of dialectic. We need only call to mind
Diderot's \emph{Le Neveu de Rameau}, and Rousseau's \emph{Discours sur l'origine
et les fondements de l'inegalite parmi less hommes}. We give here, in brief, the
essential character of these two modes of thought.

When we consider and reflect upon Nature at large, or the history of mankind, or
our own intellectual activity, at first we see the picture of an endless
entanglement of relations and reactions, permutations and combinations, in which
nothing remains what, where and as it was, but everything moves, changes, comes
into being and passes away. We see, therefore, at first the picture as a whole,
with its individual parts still more or less kept in the background; we observe
the movements, transitions, connections, rather than the things that move,
combine, and are connected. This primitive, naive but intrinsically correct
conception of the world is that of ancient Greek philosophy, and was first
clearly formulated by Heraclitus: everything is and is not, for everything is
fluid, is constantly changing, constantly coming into being and passing
away\endnote{
  Unknown to the Western world until the 20th century, the Chinese philosopher
  Lao Tzu was a predecessor of or possible contemporary to Heraclitus. Lao Tzu
  wrote the renowned Tao Te Ching in which he also espouses the fundamental
  principles of dialectics.
}.

But this conception, correctly as it expresses the general character of the
picture of appearances as a whole, does not suffice to explain the details of
which this picture is made up, and so long as we do not understand these, we
have not a clear idea of the whole picture. In order to understand these
details, we must detach them from their natural, special causes, effects, etc.
This is, primarily, the task of natural science and historical research:
branches of science which the Greek of classical times, on very good grounds,
relegated to a subordinate position, because they had first of all to collect
materials for these sciences to work upon. A certain amount of natural and
historical material must be collected before there can be any critical
analysis, comparison, and arrangement in classes, orders, and species. The
foundations of the exact natural sciences were, therefore, first worked out by
the Greeks of the Alexandrian period\endnote{
  The Alexandrian period of development of science comprises the period
  extending from the 3rd century B.C. to the 17th century A.D. It derives its
  name from the town of Alexandria in Egypt, which was one of the most
  important centres of international economic intercourses at that time. In the
  Alexandrian period, mathematics (Euclid and Archimedes), geography,
  astronomy, anatomy, physiology, etc., attained considerable development.

  China also began development in the natural sciences in the third century
  B.C.
}, and later on, in the Middle Ages, by the Arabs. Real natural science dates
from the second half of the 15th century, and thence onward it had advanced
with constantly increasing rapidity. The analysis of Nature into its individual
parts, the grouping of the different natural processes and objects in definite
classes, the study of the internal anatomy of organized bodies in their
manifold forms---these were the fundamental conditions of the gigantic strides
in our knowledge of Nature that have been made during the last 400 years. But
this method of work has also left us as legacy the habit observing natural
objects and processes in isolation, apart from their connection with the vast
whole, of observing them in repose, not in motion; as constraints, not as
essentially variables; in their death, not in their life. And when this way of
looking at things was transferred by Bacon and Locke from natural science to
philosophy, it begot the narrow, metaphysical mode of thought peculiar to the
last century.

To the metaphysician, things and their mental reflexes, ideas, are isolated, are
to be considered one after the other and apart from each other, are objects of
investigation fixed, rigid, given once for all. He thinks in absolutely
irreconcilable antitheses. His communication is His communication is ``yea, yea;
nay, nay''; for whatsoever is more than these cometh of evil. For him, a thing
either exists or does not exist; a thing cannot at the same time be itself and
something else. Positive and negative absolutely exclude one another; cause and
effect stand in a rigid antithesis, one to the other.

At first sight, this mode of thinking seems to us very luminous, because it is
that of so-called sound common sense. Only sound common sense, respectable
fellow that he is, in the homely realm of his own four walls, has very wonderful
adventures directly he ventures out into the wide world of research. And the % TODO check sentence
metaphysical mode of thought, justifiable and necessary as it is in a number of
domains whose extent varies according to the nature of the particular object of
investigation, sooner or later reaches a limit, beyond which it becomes
one-sided, restricted, abstract, lost in insoluble contradictions. In the
contemplation of individual things, it forgets the connection between them; in
the contemplation of their existence, it forgets the beginning and end of that
existence; of their repose, it forgets their motion. It cannot see the woods for
the trees.

For everyday purposes, we know and can say, e.g., whether an animal is alive or
not. But, upon closer inquiry, we find that his is, in many cases, a very
complex question, as the jurists know very well. They have cudgelled their
brains in vain to discover a rational limit beyond which the killing of the
child in its mother's womb is murder. It is just as impossible to determine
absolutely the moment of death, for physiology proves that death is not an
instantaneous, momentary phenomenon, but a very protracted process.

In like manner, every organized being is every moment the same and not the same;
every moment it assimilates matter supplied from without, and gets rid of other
matter; every moment, some cells of its body die and others build themselves
anew; in a longer or shorter time, the matter of its body is completely renewed,
and is replaced by other molecules of matter, so that every organized being is
always itself, and yet something other than itself.

Further, we find upon closer investigation that the two poles of an antithesis,
positive and negative, e.g., are as inseparable as they are opposed, and that
despite all their opposition, they mutually interpenetrate. And we find, in like
manner, that cause and effect are conceptions which only hold good in their
application to individual cases; but as soon as we consider the individual cases
in their general connection with the universe as a whole, they run into each
other, and they become confounded when we contemplate that universal action and
reaction in which causes and effects are eternally changing places, so that what
is effect here and now will be cause there and then, and vice versa.

None of these processes and modes of thought enters into the framework of
metaphysical reasoning. Dialectics, on the other hand, comprehends things and
their representations, ideas, in their essential connection, concatenation,
motion, origin and ending. Such processes as those mentioned above are,
therefore, so many corroborations of its own method of procedure.

Nature is the proof of dialectics, and it must be said for modern science that
it has furnished this proof with very rich materials increasingly daily, and
thus has shown that, in the last resort, Nature works dialectically and not
metaphysically; that she does not move in the eternal oneness of a perpetually
recurring circle, but goes through a real historical evolution. In this
connection, Darwin must be named before all others. He dealt the metaphysical
conception of Nature the heaviest blow by his proof that all organic beings,
plants, animals, and man himself, are the products of a process of evolution
going on through millions of years. But, the naturalists, who have learned to
think dialectically, are few and far between, and this conflict of the results
of discovery with preconceived modes of thinking, explains the endless confusion
now reigning in theoretical natural science, the despair of teachers as well as
learners, of authors and readers alike.

An exact representation of the universe, of its evolution, of the development of
mankind, and of the reflection of this evolution in the minds of men, can
therefore only be obtained by the methods of dialectics with its constant regard
to the innumerable actions and reactions of life and death, of progressive or
retrogressive changes. And in this spirit, the new German philosophy has worked.
Kant began his career by resolving the stable Solar system of Newton and its
eternal duration, after the famous initial impulse had once been given, into the
result of a historical process, the formation of the Sun and all the planets out
of a rotating, nebulous mass. From this, he at the same time drew the conclusion
that, given this origin of the Solar system, its future death followed of
necessity. His theory, half a century later, was established mathematically by
Laplace, and half a century after that, the spectroscope proved the existence in
space of such incandescent masses of gas in various stages of condensation.

This new German philosophy culminated in the Hegelian system. In this
system---and herein is its great merit---for the first time the whole world,
natural, historical, intellectual, is represented as a process---i.e., as in
constant motion, change, transformation, development; and the attempt is made to
trace out the internal connection that makes a continuous whole of all this
movement and development. From this point of view, the history of mankind no
longer appeared as a wild whirl of senseless deeds of violence, all equally
condemnable at the judgement seat of mature philosophic reason and which are best
forgotten as quickly as possible, but as the process of evolution of man
himself. It was now the task of the intellect to follow the gradual march of
this process through all its devious ways, and to trace out the inner law running
through all its apparently accidental phenomena.

That the Hegelian system did not solve the problem it propounded is here
immaterial. Its epoch-making merit was that it propounded the problem. This
problem is one that no single individual will ever be able to solve. Although
Hegel was---with Saint-Simon---the most encyclopedic mind of his time, yet he
was limited, first, by the necessary limited extent of his own knowledge and,
second, by the limited extent and depth of the knowledge and conceptions of his
age. To these limits, a third must be added; Hegel was an idealist. To him, the
thoughts within his brain were not the more or less abstract pictures of actual
things and processes, but, conversely, things and their evolution were only the
realized pictures of the ``Idea'', existing somewhere from eternity before the
world was. This way of thinking turned everything upside down, and completely
reversed the actual connection of things in the world. Correctly and ingeniously
as many groups of facts were grasped by Hegel, yet, for the reasons just given,
there is much that is botched, artificial, laboured, in a word, wrong in point
of detail. The Hegelian system, in itself, was a colossal miscarriage---but it
was also the last of its kind.

It was suffering, in fact, from an internal and incurable contradiction. Upon
the one hand, its essential proposition was the conception that human history is
a process of evolution, which, by its very nature, cannot find its intellectual
final term in the discovery of any so-called absolute truth. But, on the other
hand, it laid claim to being the very essence of this absolute truth. A system
of natural and historical knowledge, embracing everything, and final for all
time, is a contradiction to the fundamental law of dialectic reasoning.

This law, indeed, by no means excludes, but, on the contrary, includes the idea
that the systematic knowledge of the external universe can make giant strides
from age to age.

The perception of the fundamental contradiction in German idealism led
necessarily back to materialism, but---\emph{nota bene}---not to the simply
metaphysical, exclusively mechanical materialism of the 18th century. Old
materialism looked upon all previous history as a crude heap of irrationality
and violence; modern materialism sees in it the process of evolution of
humanity, and aims at discovering the laws thereof. With the French of the 18th
century, and even with Hegel, the conception obtained of Nature as a
whole---moving in narrow circles, and forever immutable, with its eternal
celestial bodies, as Newton, and unalterable organic species, as Linnaeus,
taught. Modern materialism embraces the more recent discoveries of natural
science, according to which Nature also has its history in time, the celestial
bodies, like the organic species that, under favourable conditions, people them,
being born and perishing. And even if Nature, as a whole, must still be said to
move in recurrent cycles, these cycles assume infinitely larger dimensions. In
both aspects, modern materialism is essentially dialectic, and no longer
requires the assistance of that sort of philosophy which, queen-like, pretended
to rule the remaining mob of sciences. As soon as each special science is bound
to make clear its position in the great totality of things and of our knowledge
of things, a special science dealing with this totality is superfluous or
unnecessary. That which still survives of all earlier philosophy is the science
of thought and its law---formal logic and dialectics. Everything else is
subsumed in the positive science of Nature and history.

Whilst, however, the revolution in the conception of Nature could only be made
in proportion to the corresponding positive materials furnished by research,
already much earlier certain historical facts had occurred which led to a
decisive change in the conception of history. In 1831, the first working-class
rising took place in Lyons; between 1838 and 1842, the first national
working-class movement, that of the English Chartists, reached its height. The
class struggle between proletariat and bourgeoisie came to the front in the
history of the most advanced countries in Europe, in proportion to the
development, upon the one hand, of modern industry, upon the other, of the
newly-acquired political supremacy of the bourgeoisie. Facts more and more
strenuously gave the lie to the teachings of bourgeois economy as to the
identity of the interests of capital and labour, as to the universal harmony and
universal prosperity that would be the consequence of unbridled competition. All
these things could no longer be ignored, any more than the French and English
Socialism, which was their theoretical, though very imperfect, expression. But
the old idealist conception of history, which was not yet dislodged, knew
nothing of class struggles based upon economic interests, knew nothing of
economic interests; production and all economic relations appeared in it only as
incidental, subordinate elements in the ``history of civilization''.

The new facts made imperative a new examination of all past history. Then it was
seen that \emph{all} past history, with the exception of its primitive stages,
was the history of class struggles; that these warring classes of society are
always the products of the modes of production and of exchange---in a word, of
the \emph{economic} conditions of their time; that the economic structure of
society always furnishes the real basis, starting from which we can alone work
out the ultimate explanation of the whole superstructure of juridical and
political institutions as well as of the religious, philosophical, and other
ideas of a given historical period. Hegel has freed history from
metaphysics---he made it dialectic; but his conception of history was
essentially idealistic. But now idealism was driven from its last refuge, the
philosophy of history; now a materialistic treatment of history was propounded,
and a method found of explaining man's ``knowing'' by his ``being'', instead of,
as heretofore, his ``being'' by his ``knowing''.

From that time forward, Socialism was no longer an accidental discovery of this
or that ingenious brain, but the necessary outcome of the struggle between two
historically developed classes---the proletariat and the bourgeoisie. Its task
was no longer to manufacture a system of society as perfect as possible, but to
examine the historico-economic succession of events from which these classes and
their antagonism had of necessity sprung, and to discover in the economic
conditions thus created the means of ending the conflict. But the Socialism of
earlier days was as incompatible with this materialist conception as the
conception of Nature of the French materialists was with dialectics and modern
natural science. The Socialism of earlier days certainly criticized the existing
capitalistic mode of production and its consequences. But it could not explain
them, and, therefore, could not get the mastery of them. It could only simply
reject them as bad. The more strongly this earlier Socialism denounced the
exploitations of the working-class, inevitable under Capitalism, the less able
was it clearly to show in what this exploitation consisted and how it arose, but
for this it was necessary---
%
\begin{itemize}
  \item{
    to present the capitalistic mode of production in its historical connection
    and its inevitableness during a particular historical period, and therefore,
    also, to present its inevitable downfall; and
  }
  \item{
    to lay bare its essential character, which was still a secret. This was done
    by the discovery of \emph{surplus-value}.
  }
\end{itemize}
%
It was shown that the appropriation of unpaid labour is the basis of the
capitalist mode of production and of the exploitation of the worker that occurs
under it; that even if the capitalist buys the labour power of his labourer at
its full value as a commodity on the market, he yet extracts more value from it
than he paid for; and that in the ultimate analysis, this surplus-value forms
those sums of value from which are heaped up constantly increasing masses of
capital in the hands of the possessing classes. The genesis of capitalist
production and the production of capital were both explained.

These two great discoveries, the materialistic conception of history and the
revelation of the secret of capitalistic production through surplus-value, we
owe to Marx. With these discoveries, Socialism became a science. The next thing
was to work out all its details and relations.

\printendnotes
