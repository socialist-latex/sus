\chapter{The Development of Utopian Socialism}

Modern Socialism is, in its essence, the direct product of the recognition, on
the one hand, of the class antagonisms existing in the society of today between
proprietors and non-proprietors, between capitalists and wage-workers; on the
other hand, of the anarchy existing in production. But, in its theoretical form,
modern Socialism originally appears ostensibly as a more logical extension of
the principles laid down by the great French philosophers of the 18th century.
Like every new theory, modern Socialism had, at first, to connect itself with
the intellectual stock-in-trade ready to its hand, however deeply its roots lay
in material economic facts.

The great men, who in France prepared men's minds for the coming revolution,
were themselves extreme revolutionists. They recognized no external authority of
any kind whatever. Religion, natural science, society, political
institutions---everything was subjected to the most unsparing criticism:
everything must justify its existence before the judgement-seat of reason or
give up existence. Reason became the sole measure of everything. It was the time
when, as Hegel says, the world stood upon its head\endnote{
  This is the passage on the French revolution:
  %
  \begin{quote}
    ``Thought, the concept of law, all at once made itself felt, and against
    this the old scaffolding of wrong could make no stand. In this conception
    of law, therefore, a constitution has now been established, and henceforth
    everything must be based upon this. Since the Sun had been in the firmament,
    and the planets circled around him, the sight had never been seen of man
    standing upon his head---i.e., on the Idea---and building reality after this
    image. Anaxagoras first said that the \emph{nous}, Reason, rules the world;
    but now, for the first time, had men come to recognize that the Idea must
    rule the mental reality. And this was a magnificent sunrise. All thinking
    Beings have participated in celebrating this holy day. A sublime emotion
    swayed men at that time, an enthusiasm of reason pervaded the world, as if
    now had come the reconciliation of the Divine Principle with the world.''
    [Hegel's \emph{The Philosophy of History}, 1840, p. 535].
  \end{quote}
  %
  Is it not high time to set the anti-Socialist law in action against such
  teachings, subversive and to the common danger, by the late Professor Hegel?
}; first in the sense that the human head, and the principles arrived at by its
thought, claimed to be the basis of all human action and association; but by
and by, also, in the wider sense that the reality which was in contradiction to
these principles had, in fact, to be turned upside down. Every form of society
and government then existing, every old traditional notion, was flung into the
lumber-room as irrational; the world had hitherto allowed itself to be led
solely by prejudices; everything in the past deserved only pity and contempt.
Now, for the first time, appeared the light of day, the kingdom of reason;
henceforth superstition, injustice, privilege, oppression, were to be superseded
by eternal truth, eternal Right, equality based on Nature and the inalienable
rights of man.

We know today that this kingdom of reason was nothing more than the idealized
kingdom of the bourgeoisie; that this eternal Right found its realization in
bourgeois justice; that this equality reduced itself to bourgeois equality
before the law; that bourgeois property was proclaimed as one of the essential
rights of man; and that the government of reason, the \emph{Contract Social} of
Rousseau, came into being, and only could come into being, as a democratic
bourgeois republic. The great thinkers of the 18th century could, no more than
their predecessors, go beyond the limits imposed upon them by their epoch.

But, side by side with the antagonisms of the feudal nobility and the burghers,
who claimed to represent all the rest of society, was the general antagonism of
exploiters and exploited, of rich idlers and poor workers. It was this very
circumstance that made it possible for the representatives of the bourgeoisie to
put themselves forward as representing not one special class, but the whole of
suffering humanity. Still further. From its origin the bourgeoisie was saddled
with its antithesis: capitalists cannot exist without wage-workers, and, in the
same proportion as the medieval burgher of the guild developed into the modern
bourgeois, the guild journeyman and the day-labourer, outside the guilds,
developed into the proletarian. And although, upon the whole, the bourgeoisie,
in their struggle with the nobility, could claim to represent at the same time
the interests of the different working-classes of that period, yet in every
great bourgeois movement there were independent outbursts of that class which
was the forerunner, more or less developed, of the modern proletariat. For
example, at the time of the German Reformation and the Peasants' War, the
Anabaptists and Thomas Müntzer; in the great English Revolution, the Levellers;
in the great French Revolution, Babeuf.

These were theoretical enunciations, corresponding with these revolutionary
uprisings of a class not yet developed; in the 16th and 17th centuries, Utopian
pictures of ideal social conditions; in the 18th century, actual communistic
theories (Morelly and Mably)\endnote{
  Engels refers here to the works of the utopian Socialists Thomas More (16th
  century) and Tommaso Campanella (17th century). See \emph{Code de la nature},
  Morelly, Paris 1841 and \emph{De le législation, ou principe des lois}, Mably,
  Amsterdam 1776.
}. The demand for equality was no longer limited to political rights; it was
extended also to the social conditions of individuals. It was not simply class
privileges that were to be abolished, but class distinctions themselves. A
Communism, ascetic, denouncing all the pleasures of life, Spartan, was the first
form of the new teaching. Then came the three great Utopians: Saint-Simon, to
whom the middle-class movement, side by side with the proletarian, still had a
certain significance; Fourier; and Owen, who in the country where capitalist
production was most developed, and under the influence of the antagonisms
begotten of this, worked out his proposals for the removal of class distinction
systematically and in direct relation to French materialism.

One thing is common to all three. Not one of them appears as a representative of
the interests of that proletariat which historical development had, in the
meantime, produced. Like the French philosophers, they do not claim to
emancipate a particular class to begin with, but all of humanity at once. Like
them, they wish to bring in the kingdom of reason and eternal justice, but this
kingdom, as they see it, is as far as Heaven from Earth, from that of the French
philosophers.

For, to our three social reformers, the bourgeois world, based upon the
principles of these philosophers, is quite as irrational and unjust, and,
therefore, finds its way to the dust-hole quite as readily as feudalism and all
the earlier stages of society. If pure reason and justice have not, hitherto,
ruled the world, this has been the case only because men have not rightly
understood them. What was wanted was the individual man of genius, who has now
arisen and who understands the truth. That he has now arisen, that the truth has
now been clearly understood, is not an inevitable event, following of necessity
in the chains of historical development, but a mere happy accident. He might
just as well have been born 500 years earlier, and might then have spared
humanity 500 years of error, strife, and suffering.

We saw how the French philosophers of the 18th century, the forerunners of the
Revolution, appealed to reason as the sole judge of all that is. A rational
government, rational society, were to be founded; everything that ran counter to
eternal reason was to be remorselessly done away with. We saw also that this
eternal reason was in reality nothing but the idealized understanding of the
18th century citizen, just then evolving into the bourgeois. The French
Revolution had realized this rational society and government.

But the new order of things, rational enough as compared with earlier
conditions, turned out to be by no means absolutely rational. The state based
upon reason completely collapsed. Rousseau's \emph{Contract Social} had found
its realization in the Reign of Terror, from which the bourgeoisie, who had lost
confidence in their own political capacity, had taken refuge first in the
corruption of the Directorate, and, finally, under the wing of the Napoleonic
despotism. The promised eternal peace was turned into an endless war of
conquest. The society based upon reason had fared no batter. The antagonism
between rich and poor, instead of dissolving into general prosperity, had become
intensified by the removal of the guild and other privileges, which had to some
extent bridged it over, and by the removal of the charitable institutions of the
Church. The ``freedom of property'' from feudal fetters, now veritably
accomplished, turned out to be, for the small capitalists and small proprietors,
the freedom to sell their small property, crushed under the overmastering
competition of the large capitalists and landlords, to these great lords, and
thus, as far as the small capitalists and peasant proprietors were concerned,
became ``freedom \emph{from} property''. The development of industry upon a
capitalistic basis made poverty and misery of the working masses conditions of
existence of society. Cash payment became more and more, in Carlyle's phrase%
\footnote{See Thomas Carlyle, \emph{Past and Present}, 1843}, the sole nexus
between man and man. The number of crimes increased from year to year. Formerly,
the feudal vices had openly stalked about in broad daylight; thought not
eradicated, they were now at any rate thrust into the background. In their
stead, the bourgeois vices, hitherto practised in secret, began to blossom all
the more luxuriantly. Trade became to a greater and greater extent cheating. The
``fraternity'' of the revolutionary motto was realized in the chicanery and
rivalries of the battle of competition. Oppression by force was replaced by
corruption; the sword, as the first social level, by gold. The right of the
first night was transferred from the feudal lords to the bourgeois
manufacturers. Prostitution increased to an extent never heard of. Marriage
itself remained, as before, the legally recognized form, the official cloak of
prostitution, and, moreover, was supplemented by rich crops of adultery.

In a word, compared with the splendid promises of the philosophers, the social
and political institutions born of the ``triumph of reason'' were bitterly
disappointing caricatures. All that was wanting was the men to formulate this
disappointment, and they came with the turn of the century. In 1802,
Saint-Simon's Geneva letters appeared; in 1808 appeared Fourier's first work,
although the groundwork of his theory dated from 1799; on January 1, 1800,
Robert Owen undertook the direction of New Lanark.

At this time, however, the capitalist mode of production, and with it the
antagonism between the bourgeoisie and the proletariat, was still very
incompletely developed. Modern Industry, which had just arisen in England, was
still unknown in France. But Modern Industry develops, on the one hand, the
conflicts which make absolutely necessary a revolution in the mode of
production, and the doing away with its capitalistic character – conflicts not
only between the classes begotten of it, but also between the very productive
forces and the forms of exchange created by it. And, on the other hand, it
develops, in these very gigantic productive forces, the means of ending these
conflicts. If, therefore, about the year 1800, the conflicts arising from the
new social order were only just beginning to take shape, this holds still more
fully as to the means of ending them. The ``have-nothing'' masses of Paris,
during the Reign of Terror, were able for a moment to gain the mastery, and thus
to lead the bourgeois revolution to victory in spite of the bourgeoisie
themselves. But, in doing so, they only proved how impossible it was for their
domination to last under the conditions then obtaining. The proletariat, which
then for the first time evolved itself from these ``have-nothing'' masses as the
nucleus of a new class, as yet quite incapable of independent political action,
appeared as an oppressed, suffering order, to whom, in its incapacity to help
itself, help could, at best, be brought in from without or down from above.

The historical situation also dominated the founders of Socialism. To the crude
conditions of capitalistic production and the crude class conditions correspond
crude theories. The solution of the social problems, which as yet lay hidden in
undeveloped economic conditions, the Utopians attempted to evolve out of the
human brain. Society presented nothing but wrongs; to remove these was the task
of reason. It was necessary, then, to discover a new and more perfect system of
social order and to impose this upon society from without by propaganda, and,
wherever it was possible, by the example of model experiments. These new social
systems were foredoomed as Utopian; the more completely they were worked out in
detail, the more they could not avoid drifting off into pure phantasies.

These facts once established, we need not dwell a moment longer upon this side
of the question, now wholly belonging to the past. We can leave it to the
literary small fry to solemnly quibble over these phantasies, which today make
us smile, and to crow over the superiority of their own bald reasoning, as
compared with such ``insanity''. For ourselves, we delight in the stupendously
grand thoughts and germs of thought that everywhere break out through their
phantastic covering, and to which these Philistines are blind.

Saint-Simon was a son of the great French Revolution, at the outbreak of which
he was not yet 30. The Revolution was the victory of the 3rd estate---i.e., of
the great masses of the nation, \emph{working} in production and in trade, over
the privileged \emph{idle} classes, the nobles and the priests. But the victory
of the 3rd estate soon revealed itself as exclusively the victory of a smaller
part of this ``estate'', as the conquest of political power by the socially
privileged section of it---i.e., the propertied bourgeoisie. And the bourgeoisie
had certainly developed rapidly during the Revolution, partly by speculation in
the lands of the nobility and of the Church, confiscated and afterwards put up
for sale, and partly by frauds upon the nation by means of army contracts. It
was the domination of these swindlers that, under the Directorate, brought
France to the verge of ruin, and thus gave Napoleon the pretext for his
\emph{coup d'état}.

Hence, to Saint-Simon the antagonism between the 3rd Estate and the privileged
classes took the form of an antagonism between ``workers'' and ``idlers''. The
idlers were not merely the old privileged classes, but also all who, without
taking any part in production or distribution, lived on their incomes. And the
workers were not only the wage-workers, but also the manufacturers, the
merchants, the bankers. That the idlers had lost the capacity for intellectual
leadership and political supremacy had been proved, and was by the Revolution
finally settled. That the non-possessing classes had not this capacity seemed to
Saint-Simon proved by the experiences of the Reign of Terror. Then, who was to
lead and command? According to Saint-Simon, science and industry, both united by
a new religious bond, destined to restore that unity of religious ideas which
had been lost since the time of the Reformation---a necessarily mystic and
rigidly hierarchic ``new Christianity''. But science, that was the scholars; and
industry, that was, in the first place, the working bourgeois, manufacturers,
merchants, bankers. These bourgeois were, certainly, intended by Saint-Simon to
transform themselves into a kind of public officials, of social trustees; but
they were still to hold, \emph{vis-à-vis} of the workers, a commanding and
economically privileged position. The bankers especially were to be called upon
to direct the whole of social production by the regulation of credit. This
conception was in exact keeping with a time in which Modern Industry in France
and, with it, the chasm between bourgeoisie and proletariat was only just coming
into existence. But what Saint-Simon especially lays stress upon is this: what
interests him first, and above all other things, is the lot of the class that is
the most numerous and the most poor (``\emph{la classe la plus nombreuse et la
plus pauvre}'').

Already in his Geneva letters, Saint-Simon lays down the proposition that ``all
men ought to work''. In the same work he recognizes also that the Reign of
Terror was the reign of the non-possessing masses.
%
\begin{quote}
  ``See'', says he to them, ``what happened in France at the time when your
  comrades held sway there; they brought about a famine''. [\emph{Lettres d’un
  habitant de Genève à ses contemporains}, Saint-Simon, 1803]
\end{quote}
%
But to recognize the French Revolution as a class war, and not simply one
between nobility and bourgeoisie, but between nobility, bourgeoisie, and the
non-possessors was, in the year 1802, a most pregnant discovery. In 1816, he
declares that politics is the science of production, and foretells the complete
absorption of politics by economics. The knowledge that economic conditions are
the basis of political institutions appears here only in embryo. Yet what is
here already very plainly expressed is the idea of the future conversion of
political rule over men into an administration of things and a direction of
processes of production---that is to say, the ``abolition of the state'', about
which recently there has been so much noise.

Saint-Simon shows the same superiority over his contemporaries, when in 1814,
immediately after the entry of the allies into Paris, and again in 1815, during
the Hundred Days' War, he proclaims the alliance of France and England, and then
of both of these countries, with Germany, as the only guarantee for the
prosperous development and peace of Europe. To preach to the French in 1815 an
alliance with the victors of Waterloo required as much courage as historical
foresight.

If in Saint-Simon we find a comprehensive breadth of view, by virtue of which
almost all the ideas of later Socialists that are not strictly economic are
found in him in embryo, we find in Fourier a criticism of the existing
conditions of society, genuinely French and witty, but not upon that account any
the less thorough. Fourier takes the bourgeoisie, their inspired prophets before
the Revolution, and their interested eulogists after it, at their own word. He
lays bare remorselessly the material and moral misery of the bourgeois world. He
confronts it with the earlier philosophers' dazzling promises of a society in
which reason alone should reign, of a civilization in which happiness should be
universal, of an illimitable human perfectibility, and with the rose-coloured
phraseology of the bourgeois ideologists of his time. He points out how
everywhere the most pitiful reality corresponds with the most high-sounding
phrases, and he overwhelms this hopeless fiasco of phrases with his mordant
sarcasm.

Fourier is not only a critic, his imperturbably serene nature makes him a
satirist, and assuredly one of the greatest satirists of all time. He depicts,
with equal power and charm, the swindling speculations that blossomed out upon
the downfall of the Revolution, and the shopkeeping spirit prevalent in, and
characteristic of, French commerce at that time. Still more masterly is his
criticism of the bourgeois form of the relations between sexes, and the position
of woman in bourgeois society. He was the first to declare that in any given
society the degree of woman’s emancipation is the natural measure of the general
emancipation. 

But Fourier is at his greatest in his conception of the history of society. He
divides its whole course, thus far, into four stages of evolution---savagery,
barbarism, the patriarchate, civilization. This last is identical with the
so-called civil, or bourgeois, society of today---i.e., with the social order
that came in with the 16th century. He proves ``that the civilized stage raises
every vice practised by barbarism in a simple fashion into a form of existence,
complex, ambiguous, equivocal, hypocritical''---that civilization moves ``in a
vicious circle'', in contradictions which it constantly reproduces without being
able to solve them; hence it constantly arrives at the very opposite to that
which it wants to attain, so that, e.g., ``under civilization poverty is born of
superabundance itself''. [\emph{Théorie de l’unite universelle}, Fourier, 1843
and \emph{Le nouveau monde industriel et sociétaire, ou invention du procédé
d'industrie attrayante et enaturelle distribuée en séries passionnées}, Fourier,
1845]

Fourier, as we see, uses the dialectic method in the same masterly way as his
contemporary, Hegel. Using these same dialectics, he argues against talk about
illimitable human perfectibility, that every historical phase has its period of
ascent and also its period of descent, and he applies this observation to the
future of the whole human race. As Kant introduced into natural science the idea
of the ultimate destruction of the Earth, Fourier introduced into historical
science that of the ultimate destruction of the human race.

Whilst in France the hurricane of the Revolution swept over the land, in England
a quieter, but not on that account less tremendous, revolution was going on.
Steam and the new tool-making machinery were transforming manufacture into
modern industry, and thus revolutionizing the whole foundation of bourgeois
society. The sluggish march of development of the manufacturing period changed
into a veritable storm and stress period of production. With constantly
increasing swiftness the splitting-up into large capitalists and non-possessing
proletarians went on. Between these, instead of the former stable middle-class,
an unstable mass of artisans and small shopkeepers, the most fluctuating portion
of the population, now led a precarious existence.

The new mode of production was, as yet, only at the beginning of its period of
ascent; as yet it was the normal, regular method of production---the only one
possible under existing conditions. Nevertheless, even then it was producing
crying social abuses---the herding together of a homeless population in the
worst quarters of the large towns; the loosening of all traditional moral bonds,
of patriarchal subordination, of family relations; overwork, especially of women
and children, to a frightful extent; complete demoralization of the
working-class, suddenly flung into altogether new conditions, from the country
into the town, from agriculture into modern industry, from stable conditions of
existence into insecure ones that change from day to day.

At this juncture, there came forward as a reformer a manufacturer
29-years-old---a man of almost sublime, childlike simplicity of character, and
at the same time one of the few born leaders of men. Robert Owen had adopted the
teaching of the materialistic philosophers: that man's character is the product,
on the one hand, of heredity; on the other, of the environment of the individual
during his lifetime, and especially during his period of development. In the
industrial revolution most of the class saw only chaos and confusion, and the
opportunity of fishing in these troubled waters and making large fortunes
quickly. He saw in it the opportunity of putting into practice his favourite
theory, and so of bringing order out of chaos. He had already tried it with
success, as superintendent of more than 500 men in a Manchester factory. From
1800 to 1829, he directed the great cotton mill at New Lanark, in Scotland, as
managing partner, along the same lines, but with greater freedom of action and
with a success that made him a European reputation. A population, originally
consisting of the most diverse and, for the most part, very demoralized
elements, a population that gradually grew to 2\,500, he turned into a model
colony, in which drunkenness, police, magistrates, lawsuits, poor laws, charity,
were unknown. And all this simply by placing the people in conditions of worthy
human beings, and especially by carefully bringing up the rising generation. He
was the founder of infant schools, and introduced them first at New Lanark. At
the age of two, the children came to school, where they enjoyed themselves so
much that they could scarcely be got home again. Whilst his competitors worked
their people 13 or 14 hours a day, in New Lanark the working-day was only 10 and
a half hours. When a crisis in cotton stopped work for four months, his workers
received their full wages all the time. And with all this the business more than
doubled in value, and to the last yielded large profits to its proprietors.

In spite of all this, Owen was not content. The existence which he secured for
his workers was, in his eyes, still far from being worthy of human beings. ``The
people were slaves at my mercy.'' The relatively favourable conditions in which
he had placed them were still far from allowing a rational development of the
character and of the intellect in all directions, much less of the free exercise
of all their faculties.
%
\begin{quote}
  ``And yet, the working part of this population of 2\,500 persons was daily
  producing as much real wealth for society as, less than half a century
  before, it would have required the working part of a population of 600\,000
  to create. I asked myself, what became of the difference between the wealth
  consumed by 2\,500 persons and that which would have been consumed by
  600\,000?''\endnote{
    From \emph{The Revolution in Mind and Practice}, p. 21, a memorial
    addressed to all the ``red Republcians, Communists and Socialists of
    Europe'', and sent to the provisional government of France, 1848, and also
    ``to Queen Victoria and her responsible advisers''.
  }
\end{quote}
%
The answer was clear. It had been used to pay the proprietors of the
establishment 5\% on the capital they had laid out, in addition to over
£300\,000 clear profit. And that which held for New Lanark held to a still
greater extent for all the factories in England.
%
\begin{quote}
  ``If this new wealth had not been created by machinery, imperfectly as it has
  been applied, the wars of Europe, in opposition to Napoleon, and to support
  the aristocratic principles of society, could not have been maintained. And
  yet this new power was the creation of the working-classes.'' % TODO interpret source
\end{quote}
%
To them, therefore, the fruits of this new power belonged. The newly-created
gigantic productive forces, hitherto used only to enrich individuals and to
enslave the masses, offered to Owen the foundations for a reconstruction of
society; they were destined, as the common property of all, to be worked for the
common good of all.

Owen's communism was based upon this purely business foundation, the outcome, so
to say, of commercial calculation. Throughout, it maintained this practical
character. Thus, in 1823, Owen proposed the relief of the distress in Ireland by
Communist colonies, and drew up complete estimates of costs of founding them,
yearly expenditure, and probable revenue. And in his definite plan for the
future, the technical working out of details is managed with such practical
knowledge---ground plan, front and side and bird's-eye views all
included---that the Owen method of social reform once accepted, there is from
the practical point of view little to be said against the actual arrangement of
details.

His advance in the direction of Communism was the turning-point in Owen’s life.
As long as he was simply a philanthropist, he was rewarded with nothing but
wealth, applause, honour, and glory. He was the most popular man in Europe. Not
only men of his own class, but statesmen and princes listened to him
approvingly. But when he came out with his Communist theories that was quite
another thing. Three great obstacles seemed to him especially to block the path
to social reform: private property, religion, the present form of marriage.

He knew what confronted him if he attacked these---outlawry, excommunication
from official society, the loss of his whole social position. But nothing of
this prevented him from attacking them without fear of consequences, and what he
had foreseen happened. Banished from official society, with a conspiracy of
silence against him in the press, ruined by his unsuccessful Communist
experiments in America, in which he sacrificed all his fortune, he turned
directly to the working-class and continued working in their midst for 30 years.

Every social movement, every real advance in England on behalf of the workers
links itself on to the name of Robert Owen. He forced through in 1819, after
five years' fighting, the first law limiting the hours of labour of women and
children in factories. He was president of the first Congress at which all the
Trade Unions of England united in a single great trade association. He
introduced as transition measures to the complete communistic organization of
society, on the one hand, cooperative societies for retail trade and production.
These have since that time, at least, given practical proof that the merchant
and the manufacturer are socially quite unnecessary. On the other hand, he
introduced labour bazaars for the exchange of the products of labour through the
medium of labour-notes, whose unit was a single hour of work; institutions
necessarily doomed to failure, but completely anticipating Proudhon's bank of
exchange of a much later period, and differing entirely from this in that it did
not claim to be the panacea for all social ills, but only a first step towards a
much more radical revolution of society.

The Utopians' mode of thought has for a long time governed the Socialist ideas
of the 19th century, and still governs some of them. Until very recently, all
French and English Socialists did homage to it. The earlier German Communism,
including that of Weitling, was of the same school. To all these, Socialism is
the expression of absolute truth, reason and justice, and has only to be
discovered to conquer all the world by virtue of its own power. And as an
absolute truth is independent of time, space, and of all the historical
development of man, it is a mere accident when and where it is discovered. With
all this, absolute truth, reason, and justice are different with the founder of
each different school. And as each one's special kind of absolute truth, reason,
and justice is again conditioned by his subjective understanding, his conditions
of existence, the measure of his knowledge and his intellectual training, there
is no other ending possible in this conflict of absolute truths than that they
shall be mutually exclusive of one another. Hence, from this nothing could come
but a kind of eclectic, average Socialism, which, as a matter of fact, has up to
the present time dominated the minds of most of the socialist workers in France
and England. Hence, a mish-mash allowing of the most manifold shades of opinion:
a mish-mash of such critical statements, economic theories, pictures of future
society by the founders of different sects, as excite a minimum of opposition;
a mish-mash which is the more easily brewed the more definite sharp edges of the
individual constituents are rubbed down in the stream of debate, like rounded
pebbles in a brook.

To make a science of Socialism, it had first to be placed upon a real basis.

\printendnotes