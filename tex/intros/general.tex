\chapter{Introductions}

\section*{General Introduction and the History of Materialism}
\addcontentsline{toc}{section}{
  General Introduction and the History of Materialism
}

The present little book is, originally, part of a larger whole. About 1875,
Dr. E. Dühring, privatdocent\footnote{University lecturer who formerly received
fees from his students rather than a wage.} at Berlin University, suddenly and
rather clamorously announced his conversion to Socialism, and presented the
German public not only with an elaborate Socialist theory, but also with a
complete practical plan for the reorganization of society. As a matter of
course, he fell foul of his predecessors; above all, he honoured Marx by pouring
out upon him the full vials of his wrath.

This took place about the same time when the two sections of the Socialist
party in Germany---Eisenachers and Lasselleans---had just effected their
fusion\footnote{At the Gotha Unification Congress.}, and thus obtained not only
an immense increase of strength, but, was what more, the faculty of employing
the whole of this strength against the common enemy. The Socialist party in
Germany was fast becoming a power. But, to make it a power, the first condition
was that the newly-conquered unity should not be imperilled. And Dr. Dühring
openly proceeded to form around himself a sect, the nucleus of a future
separate party. It, thus, became necessary to take up the gauntlet thrown
down to us, and to fight out the struggle, whether we liked it or not.

This, however, though it might not be an over-difficult, was evidently a
long-winded business. As is well-known, we Germans are of a terribly ponderous
Grundlichkeit, radical profundity or profound radicality, whatever you may like
to call it. Whenever anyone of us expounds what he considers a new doctrine,
he has first to elaborate it into an all-comprising system. He has to prove
that both the first principles of logic and the fundamental laws of the
universe had existed from all eternity for no other purpose than to
ultimately lead to this newly-discovered, crowning theory. And Dr. Dühring,
in this respect, was quite up to the national mark. Nothing less than a
complete ``System of Philosophy'', mental, moral, natural, and historical; a
complete ``System of Political Economy and Socialism''; and, finally, a
``Critical History of Political Economy''---three big volumes in octavo, heavy
extrinsically and intrinsically, three army-corps of arguments mobilized
against all previous philosophers and economists in general, and against Marx
in particular---in fact, an attempt at a complete ``revolution in
science''---these were what I should have to tackle. I had to treat of all and
every possible subject, from concepts of time and space to Bimetallism; from the
eternity of matter and motion, to the perishable nature of moral ideas; from
Darwin's natural selection to the education of youth in a future society.
Anyhow, the systematic comprehensiveness of my opponent gave me the opportunity
of developing, in opposition to him, and in a more connected form than had
previously been done, the views held by Marx and myself on this great variety
of subjects. And that was the principal reason which made me undertake this
otherwise ungrateful task.

My reply was first published in a series of articles in the Leipzig Vorwarts,
the chief organ of the Socialist party\footnote{
  Vorwarts existed in Leipzig from 1876--78, after the Gotha Unification
  Congress.
}, and later on as a book: ``Herr Eugen Dührings Umwalzung der Wissenchaft''
(Mr. E. Dühring's ``Revolution in Science''), a second edition of which appeared
in Zurich, 1886.

At the request of my friend, Paul Lafargue, now representative of Lille in the
French Chamber of Deputies, I arranged three chapters of this book as a
pamphlet, which he translated and published in 1880, under the title:
``Socialisme utopique et Socialisme scientifique''. From this French text, a
Polish and a Spanish edition were prepared. In 1883, our German friends
brought out the pamphlet in the original language. Italian, Russian, Danish,
Dutch, and Romanian translations, based upon the German text, have since been
published. Thus, the present English edition, this little book circulates in
10 languages. I am not aware that any other Socialist work, not even our
\emph{Communist Manifesto} of 1848, or Marx's \emph{Capital}, has been so often
translated. In Germany, it has had four editions of about 20\,000 copies in all.

%TODO Check rest
The Appendix, ``The Mark'', was written with the intention of spreading among
the German Socialist party some elementary knowledge of the history and
development of landed property in Germany. This seemed all the more necessary
at a time when the assimilation by that party of the working-people of the
towns was in a fair way of completion, and when the agricultural labourers and
peasant had to be taken in hand. This appendix has been included in the
translation, as the original forms of tenure of land common to all Teutonic
tribes, and the history of their decay, are even less known in England and in
Germany. I have left the text as it stands in the original, without alluding
to the hypothesis recently started by Maxim Kovalevsky, according to which the
partition of the arable and meadow lands among the members of the Mark was
preceded by their being cultivated for joint-account by a large patriarchal
family community, embracing several generations (as exemplified by the still
existing South Slavonian Zadruga), and that the partition, later on, took
place when the community had increased, so as to become too unwieldy for
joint-account management. Kovalevsky is probably quite right, but the matter is
still \emph{sub judice}\footnote{under consideration}.

The economic terms used in this work, as afar as they are new, agree with
those used in the English edition of Marx's \emph{Capital}. We call ``production
of commodities'' that economic phase where articles are produced not only for
the use of the producers, but also for the purpose of exchange; that is,
\emph{as commodities}, not as use values. This phase extends from the first
beginnings of production for exchange down to our present time; it attains its
full development under capitalist production only, that is, under conditions
where the capitalist, the owner of the means of production, employs, for wages,
labourers, people deprived of all means of production except their own
labour-power, and pockets the excess of the selling price of the products over
his outlay. We divide the history of industrial production since the Middle
Ages into three periods:
%
\begin{itemize}
  \item{
    handicraft, small master craftsman with a few journeymen and apprentices,
    where each labourer produces a complete article;
  }
  \item{
    manufacture, where greater numbers of workmen, grouped in one large
    establishment, produce the complete article on the principle of division of
    labour, each workman performing only one partial operation, so that the
    product is complete only after having passed successively through the hands
    of all;
  }
  \item{
    modern industry, where the product is produced by machinery driven by power,
    and where the work of the labourer is limited to superintending and
    correcting the performance of the mechanical agent.
  }
\end{itemize}
%
I am perfectly aware that the contents of this work will meet with objection
from a considerable portion of the British public. But, if we Continentals had
taken the slightest notice of the prejudices of British ``respectability'', we
should be even worse off than we are. This book defends what we call
``historical materialism'', and the word materialism grates upon the ears of the
immense majority of British readers. ``Agnosticism'' might be tolerated, but
materialism is utterly inadmissible.

And, yet, the original home of all modern materialism, from the 17th century
onwards, is England.
%
\begin{quote}
  ``Materialism is the natural-born son of Great Britain. Already the British
  schoolman, Duns Scotus, asked, `whether it was impossible for the matter to
  think?'

  ``The real progenitor of English materialism is Bacon. To him, natural
  philosophy is the only true philosophy, and physics based upon the
  experience of the senses is the chiefest part of natural philosophy.
  Anaxagoras and his homoiomeriae, Democritus and his atoms, he often quotes
  as his authorities. According to him, the senses are infallible and the
  source of all knowledge. All science is based on experience, and consists in
  subjecting the data furnished by the senses to a rational method of
  investigation. Induction, analysis, comparison, observation, experiment, are
  the principal forms of such a rational method. Among the qualities inherent
  in matter, motion is the first and foremost, not only in the form of
  mechanical and mathematical motion, but chiefly in the form of an impulse, a
  vital spirit, a tension — or a `qual', to use a term of Jakob Bohme's\endnote{
    ``Qual'' is a philosophical play upon words. Qual literally means torture, a
    pain which drives to action of some kind; at the same time, the mystic
    Bohme puts into the German word something of the meaning of the \emph{Latin}
    \emph{qualitas}; his ``qual'' was the activating principle arising from, and
    promoting in its turn, the spontaneous development of the thing, relation,
    or person subject to it, in contradistinction to a pain inflicted from
    without. [Note by Engels to the English Edition]
  }---of matter.

  ``In Bacon, its first creator, materialism still occludes within itself the
  germs of a many-sided development. On the one hand, matter, surrounded by a
  sensuous, poetic glamor, seems to attract man's whole entity by winning
  smiles. On the other, the aphoristically formulated doctrine pullulates with
  inconsistencies imported from theology.

  ``In its further evolution, materialism becomes one-sided. Hobbes is the man
  who systematizes Baconian materialism. Knowledge based upon the senses loses
  its poetic blossom, it passes into the abstract experience of the
  mathematician; geometry is proclaimed as the queen of sciences. Materialism
  takes to misanthropy. If it is to overcome its opponent, misanthropic,
  flashless spiritualism, and that on the latter's own ground, materialism has
  to chastise its own flesh and turn ascetic. Thus, from a sensual, it passes
  into an intellectual, entity; but thus, too, it evolves all the consistency,
  regardless of consequences, characteristic of the intellect.

  ``Hobbes, as Bacon's continuator, argues thus: if all human knowledge is
  furnished by the senses, then our concepts and ideas are but the phantoms,
  divested of their sensual forms, of the real world. Philosophy can but give
  names to these phantoms. One name may be applied to more than one of them.
  There may even be names of names. It would imply a contradiction if, on the
  one hand, we maintained that all ideas had their origin in the world of
  sensation, and, on the other, that a word was more than a word; that,
  besides the beings known to us by our senses, beings which are one and all
  individuals, there existed also beings of a general, not individual, nature.
  An unbodily substance is the same absurdity as an unbodily body. Body, being,
  substance, are but different terms for the same reality. \emph{It is
  impossible to separate thought from matter that thinks}. This matter is the
  substratum of all changes going on in the world. The word infinite is
  meaningless, unless it states that our mind is capable of performing an
  endless process of addition. Only material things being perceptible to us, we
  cannot know anything about the existence of God. My own existence alone is
  certain. Every human passion is a mechanical movement, which has a beginning
  and an end. The objects of impulse are what we call good. Man is subject to
  the same laws as nature. Power and freedom are identical.

  ``Hobbes had systematized Bacon, without, however, furnishing a proof for
  Bacon's fundamental principle, the origin of all human knowledge from the
  world of sensation. It was Locke who, in his \emph{Essay on the Human
  Understanding}, supplied this proof.

  ``Hobbes had shattered the theistic prejudices of Baconian materialism;
  Collins, Dodwell, Coward, Hartley, Priestley, similarly shattered the last
  theological bars that still hemmed in Locke's sensationalism. At all events,
  for practical materialists, Deism is but an easy-going way of getting rid of
  religion.'' [Karl Marx, \emph{The Holy Family}, p. 201--304]
\end{quote}
%
Thus Karl Marx wrote about the British origin of modern materialism. If
Englishmen nowadays do not exactly relish the compliment he paid their
ancestors, more's the pity. It is none the less undeniable that Bacon, Hobbes,
and Locke are the fathers of that brilliant school of French materialism which
made the 18th century, in spite of all battles on land and sea won over
Frenchmen by Germans and Englishmen, a pre-eminently French century, even
before that crowning French Revolution, the results of which we outsiders, in
England as well as Germany, are still trying to acclimatize.

There is no denying it. About the middle of this century, what struck every
cultivated foreigner who set up his residence in England, was what he was then
bound to consider the religious bigotry and stupidity of the English
respectable middle-class. We, at that time, were all materialists, or, at
least, very advanced free-thinkers, and to us it appeared inconceivable that
almost all educated people in England should believe in all sorts of
impossible miracles, and that even geologists like Buckland and Mantell should
contort the facts of their science so as not to clash too much with the myths
of the book of Genesis; while, in order to find people who dared to use their
own intellectual faculties with regard to religious matters, you had to go
amongst the uneducated, the ``great unwashed'', as they were then called, the
working people, especially the Owenite Socialists.

But England has been ``civilized'' since then. The exhibition of 1851 sounded
the knell of English insular exclusiveness. England became gradually
internationalized, in diet, in manners, in ideas; so much so that I begin to
wish that some English manners and customs had made as much headway on the
Continent as other Continental habits have made here. Anyhow, the introduction
and spread of salad-oil (before 1851 known only to the aristocracy) has been
accompanied by a fatal spread of Continental scepticism in matters religious,
and it has come to this, that agnosticism, though not yet considered ``the
thing'' quite as much as the Church of England, is yet very nearly on a par, as
far as respectability goes, with Baptism, and decidedly ranks above the
Salvation Army. And I cannot help believing that under those circumstances it
will be consoling to many who sincerely regret and condemn this progress of
infidelity to learn that these ``new-fangled notions'' are not of foreign
origin, are not ``made in Germany'', like so many other articles of daily use,
but are undoubtedly Old English, and that their British originators 200 years
ago went a good deal further than their descendants now dare to venture.

What, indeed, is agnosticism but, to use an expressive Lancashire term,
``shamefaced'' materialism? The agnostic's conception of Nature is materialistic
throughout. The entire natural world is governed by law, and absolutely
excludes the intervention of action from without. But, he adds, we have no
means either of ascertaining or of disproving the existence of some Supreme
Being beyond the known universe. Now, this might hold good at the time when
Laplace, to Napoleon's question, why, in the great astronomer's \emph{Treatise
on Celestial Mechanics}, the Creator was not even mentioned, proudly replied ``I
had no need of this hypothesis''. But, nowadays, in our evolutionary conception
of the universe, there is absolutely no room for either a Creator or a Ruler;
and to talk of a Supreme Being shut out from the whole existing world, implies
a contradiction in terms, and, as it seems to me, a gratuitous insult to the
feelings of religious people.

Again, our agnostic admits that all our knowledge is based upon the
information imparted to us by our senses. But, he adds, how do we know that
our senses give us correct representations of the objects we perceive through
them? And he proceeds to inform us that, whenever we speak of objects, or
their qualities, of which he cannot know anything for certain, but merely the
impressions which they have produced on his senses. Now, this line of
reasoning seems undoubtedly hard to beat by mere argumentation. But before
there was argumentation, there was action. \emph{Im Anfang war die
That}\footnote{From Goethe's \emph{Faust}: ``In the beginning was the deed''.}.
And human action had solved the difficulty long before human ingenuity invented
it. The proof of the pudding is in the eating. From the moment we turn to our
own use these objects, according to the qualities we perceive in them, we put
to an infallible test the correctness or otherwise of our sense-perception. If
these perceptions have been wrong, then our estimate of the use to which an
object can be turned must also be wrong, and our attempt must fail. But, if we
succeed in accomplishing our aim, if we find that the object does agree with our
idea of it, and does answer the purpose we intended it for, then that is proof
positive that our perceptions of it and of its qualities, \emph{so far}, agree
with reality outside ourselves. And, whenever we find ourselves face-to-face
with a failure, then we generally are not long in making out the cause that made
us fail; we find that the perception upon which we acted was either incomplete
and superficial, or combined with the results of other perceptions in a way
not warranted by them — what we call defective reasoning. So long as we take
care to train our senses properly, and to keep our action within the limits
prescribed by perceptions properly made and properly used, so long as we shall
find that the result of our action proves the conformity of our perceptions
with the objective nature of the things perceived. Not in one single instance,
so far, have we been led to the conclusion that our sense-perception,
scientifically controlled, induce in our minds ideas respecting the outer world
that are, by their very nature, at variance with reality, or that there is an
inherent incompatibility between the outer world and our sense-perceptions of
it.

But then come the Neo-Kantian agnostics and say: We may correctly perceive the
qualities of a thing, but we cannot by any sensible or mental process grasp
the thing-in-itself. This ``thing-in-itself'' is beyond our ken. To this Hegel,
long since, has replied: If you know all the qualities of a thing, you know
the thing itself; nothing remains but the fact that the said thing exists
without us; and, when your senses have taught you that fact, you have grasped
the last remnant of the thing-in-itself, Kant's celebrated unknowable \emph{Ding
an sich}. To which it may be added that in Kant's time our knowledge of natural
objects was indeed so fragmentary that he might well suspect, behind the
little we knew about each of them, a mysterious ``thing-in-itself''. But one
after another these ungraspable things have been grasped, analyzed, and, what
is more, \emph{reproduced} by the giant progress of science; and what we can
produce we certainly cannot consider as unknowable. To the chemistry of the
first half of this century, organic substances were such mysterious object; now
we learn to build them up one after another from their chemical elements without
the aid of organic processes. Modern chemists declare that as soon as the
chemical constitution of no-matter-what body is known, it can be built up from
its elements. We are still far from knowing the constitution of the highest
organic substances, the albuminous bodies; but there is no reason why we should
not, if only after centuries, arrive at the knowledge and, armed with it,
produce artificial albumen. But, if we arrive at that, we shall at the same
time have produced organic life, for life, from its lowest to its highest
forms, is but the normal mode of existence of albuminous bodies.

As soon, however, as our agnostic has made these formal mental reservations, he
talks and acts as the rank materialist he at bottom is. He may say that, as far
as \emph{we} know, matter and motion, or as it is now called, energy, can
neither be created nor destroyed, but that we have no proof of their not having
been created at some time or other. But if you try to use this admission against
him in any particular case, he will quickly put you out of court. If he admits
the possibility of spiritualism \emph{in abstracto}, he will have none of it
\emph{in concreto}. As far as we know and can know, he will tell you there is no
creator and no Ruler of the universe; as far as we are concerned, matter and
energy can neither be created nor annihilated; for us, mind is a mode of energy,
a function of the brain; all we know is that the material world is governed by
immutable laws, and so forth. Thus, as far as he is a scientific man, as far as
he \emph{knows} anything, he is a materialist; outside his science, in spheres
about which he knows nothing, he translates his ignorance into Greek and calls
it agnosticism.

At all events, one thing seems clear: even if I was an agnostic, it is evident
that I could not describe the conception of history sketched out in this little
book as ``historical agnosticism''. Religious people would laugh at me,
agnostics would indignantly ask, was I making fun of them? And, thus, I hope
even British respectability will not be overshocked if I use, in English as
well as in so many other languages, the term ``historical materialism'', to
designate that view of the course of history which seeks the ultimate cause and
the great moving power of all important historic events in the economic
development of society, in the changes in the modes of production and exchange,
in the consequent division of society into distinct classes, and in the
struggles of these classes against one another.

This indulgence will, perhaps, be accorded to me all the sooner if I show that
historical materialism may be of advantage even to British respectability. I
have mentioned the fact that, about 40 or 50 years ago, any cultivated
foreigner settling in England was struck by what he was then bound to consider
the religious bigotry and stupidity of the English respectable middle-class. I
am now going to prove that the respectable English middle-class of that time
was not quite as stupid as it looked to the intelligent foreigner. Its
religious leanings can be explained.

\printendnotes
