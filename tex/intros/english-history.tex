\section*{History of the English middle-class}
\addcontentsline{toc}{section}{History of the English middle-class}

When Europe emerged from the Middle Ages, the rising middle-class of the towns
constituted its revolutionary element. It had conquered a recognized position
within medieval feudal organization, but this position, also, had become too
narrow for its expansive power. The development of the middle-class, the
\emph{bourgeoisie}, became incompatible with the maintenance of the feudal
system; the feudal system, therefore, had to fall.

But the great international centre of feudalism was the Roman Catholic Church.
It united the whole of feudalized Western Europe, in spite of all internal wars,
into one grand political system, opposed as much to the schismatic Greeks as to
the Mohammedan countries. It had organized its own hierarchy on the feudal
model, and, lastly, it was itself by far the most powerful feudal lord, holding,
as it did, fully one third of the soil of the Catholic world. Before profane
feudalism could be successfully attacked in each country and in detail, this,
its sacred central organization, had to be destroyed.

Moreover, parallel with the rise of the middle-class went on the great revival
of science; astronomy, mechanics, physics, anatomy, physiology were again
cultivated. And the bourgeoisie, for the development of its industrial
production, required a science which ascertained the physical properties of
natural objects and the modes of action of the forces of Nature. Now up to then
science had but been the humble handmaid of the Church, had not been allowed to
overlap the limits set by faith, and for that reason had been no science at all.
Science rebelled against the Church; the bourgeoisie could not do without
science, and, therefore, had to join in the rebellion.

The above, though touching but two of the points where the rising middle-class
was bound to come into collision with the established religion, will be
sufficient to show, first, that the class most directly interested in the
struggle against the pretensions of the Roman Church was the bourgeoisie; and
second, that every struggle against feudalism, at that time, had to take on a
religious disguise, had to be directed against the Church in the first instance.
But if the universities and the traders of the cities started the cry, it was
sure to find, and did find, a strong echo in the masses of the country people,
the peasants, who everywhere had to struggle for their very existence with their
feudal lords, spiritual and temporal.

The long fight of the bourgeoisie against feudalism culminated in three great,
decisive battles.

The first was what is called the Protestant Reformation in Germany. The war cry
raised against the Church, by Luther, was responded to by two insurrections of a
political nature; first, that of the lower nobility under Franz von Sickingen
(1523), then the great Peasants' War, 1525. Both were defeated, chiefly in
consequence of the indecision of the parties most interested, the burghers of
the towns---an indecision into the causes of which we cannot here enter. From
that moment, the struggle degenerated into a fight between the local princes and
the central power, and ended by blotting out Germany, for 200 years, from the
politically active nations of Europe. The Lutheran Reformation produced a new
creed indeed, a religion adapted to absolute monarchy. No sooner were the
peasant of North-east Germany converted to Lutheranism than they were from
freemen reduced to serfs. 

But where Luther failed, Calvin won the day. Calvin's creed was one fit for the
boldest of the bourgeoisie of his time. His predestination doctrine was the
religious expression of the fact that in the commercial world of competition
success or failure does not depend upon a man's activity or cleverness, but upon
circumstances uncontrollable by him. It is not of him that willeth or of him
that runneth, but of the mercy of unknown superior economic powers; and this was
especially true at a period of economic revolution, when all old commercial
routes and centres were replaced by new ones, when India and America were opened
to the world, and when even the most sacred economic articles of faith---the
value of gold and silver---began to totter and to break down. Calvin's church
constitution of God was republicanized, could the kingdoms of this world remain
subject to monarchs, bishops, and lords? While German Lutheranism became a
willing tool in the hands of princes, Calvinism founded a republic in Holland,
and active republican parties in England, and, above all, Scotland.

In Calvinism, the second great bourgeois upheavel found its doctrine ready cut
and dried. This upheavel took place in England. The middle-class of the towns
brought it on, and the yeomanry of the country districts fought it out.
Curiously enough, in all the three great bourgeois risings, the peasantry
furnishes the army that has to do the fighting; and the peasantry is just the
class that, the victory once gained, is more surely ruined by the economic
consequences of that victory. A hundred years after Cromwell, the yeomanry of
England had almost disappeared. Anyhow, had it not been for that yeomanry and
for the \emph{plebian} element in the towns, the bourgeoisie alone would never
have fought the matter out to the bitter end, and would never have brought
Charles I to the scaffold. In order to secure even those conquests of the
bourgeoisie that were ripe for gathering at the time, the revolution had to be
carried considerably further---exactly as in 1793 in France and 1848 in Germany.
This seems, in fact, to be one of the laws of evolution of bourgeois society.

Well, upon this excess of revolutionary activity there necessarily followed the
inevitable reaction which, in its turn, went beyond the point where it might
have maintained itself. After a series of oscillations, the new centre of
gravity was at last attained and became a new starting-point. The grand period
of English history, known to respactability under the name of ``the Great
Rebellion'', and the struggles succeeding it, were brought to a close by the
comparatively puny events entitled by the Liberal historians ``the Glorious
Revolution''.

The new starting-point was a compromise between the rising middle-class and the
ex-feudal landowners. The latter, though called, as now, the aristocracy, had
been long since on the way which led them to become what Louis Philippe in
France became at a much later period: ``The first bourgeois of the kingdom''.
Fortunately for England, the old feudal barons had hilled one another during the
War of the Roses. Their successors, though mostly scions of the old families,
had been so much out of the direct line of descent that they constituted quite a
new body, with habits and tendencies far mor bourgeois than feudal. They fully
understood the value of money, and at once began to increase their rents by
turning hundreds of small farmers out and replacing them with sheep. Henry VIII,
while squandering the Church lands, created fresh bourgeois landlords by
wholesale; the innumerable confiscation of estates, regranted to absolute or
relative upstarts, and continued during the whole of the 17th century, had the
same result. Consequently, ever since Henry VII, the English ``aristocracy'',
far from counteracting the development of industrial production, had, on the
contrary, sought to indirectly profit thereby; and there had always been a
section of the great landowners willing, from economical or political reasons,
to cooperate with the leading men of the financial and industrial bourgeoisie.
The compromise of 1689 was, therefore, easily accomplished. The political spoils
of ``pelf and place'' were left to the great landowning families, provided the
economic interests of the financial, manufacturing, and commercial middle-class
were sufficiently attended to. And these economic interests were at that time
powerful enough to determine the general policy of the nation. There might be
squabbles about matters of detail, but, on the whole, the aristocratic oligarchy
knew too well that its own economic prosperity was irretrievably bound up with
that of the industrial and commercial middle-class.

From that time, the bourgeoisie was a humble, but still a recognized, component
of the ruling classes of England. With the rest of them, it had a common
interest in keeping in subjection the great working mass of the nation. The
merchant or manufacturer himself stood in the position of master, or, as it was
until lately called, of "natural superior" to his clerks, his work-people, his
domestic servants. His interest was to get as much and as good work out of them
as he could; for this end, they had to be trained to proper submission. He was
himself religious; his religion had supplied the standard under which he had
fought the king and the lords; he was not long in discovering the opportunities
this same religion offered him for working upon the minds of his natural
inferiors, and making them submissive to the behests of the masters it had
pleased God to place over them. In short, the English bourgeoisie now had to
take a part in keeping down the ``lower orders'', the great producing mass of
the nation, and one of the means employed for that purpose was the influence of
religion. 

There was another factor that contributed to strengthen the religious leanings
of the bourgeoisie. That was the rise of materialism in England. This new
doctrine not only shocked the pious feelings of the middle-class; it announced
itself as a philosophy only fit for scholars and cultivated men of the world,
in contrast to religion, which was good enough for the uneducated masses,
including the bourgeoisie.
