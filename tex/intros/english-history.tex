\section*{History of the English middle-class}
\addcontentsline{toc}{section}{History of the English middle-class}

When Europe emerged from the Middle Ages, the rising middle-class of the towns
constituted its revolutionary element. It had conquered a recognized position
within medieval feudal organization, but this position, also, had become too
narrow for its expansive power. The development of the middle-class, the
\emph{bourgeoisie}, became incompatible with the maintenance of the feudal
system; the feudal system, therefore, had to fall.

But the great international centre of feudalism was the Roman Catholic Church.
It united the whole of feudalized Western Europe, in spite of all internal wars,
into one grand political system, opposed as much to the schismatic Greeks as to
the Mohammedan countries. It had organized its own hierarchy on the feudal
model, and, lastly, it was itself by far the most powerful feudal lord, holding,
as it did, fully one third of the soil of the Catholic world. Before profane
feudalism could be successfully attacked in each country and in detail, this,
its sacred central organization, had to be destroyed.

Moreover, parallel with the rise of the middle-class went on the great revival
of science; astronomy, mechanics, physics, anatomy, physiology were again
cultivated. And the bourgeoisie, for the development of its industrial
production, required a science which ascertained the physical properties of
natural objects and the modes of action of the forces of Nature. Now up to then
science had but been the humble handmaid of the Church, had not been allowed to
overlap the limits set by faith, and for that reason had been no science at all.
Science rebelled against the Church; the bourgeoisie could not do without
science, and, therefore, had to join in the rebellion.

The above, though touching but two of the points where the rising middle-class
was bound to come into collision with the established religion, will be
sufficient to show, first, that the class most directly interested in the
struggle against the pretensions of the Roman Church was the bourgeoisie; and
second, that every struggle against feudalism, at that time, had to take on a
religious disguise, had to be directed against the Church in the first instance.
But if the universities and the traders of the cities started the cry, it was
sure to find, and did find, a strong echo in the masses of the country people,
the peasants, who everywhere had to struggle for their very existence with their
feudal lords, spiritual and temporal.

The long fight of the bourgeoisie against feudalism culminated in three great,
decisive battles.

The first was what is called the Protestant Reformation in Germany. The war cry
raised against the Church, by Luther, was responded to by two insurrections of a
political nature; first, that of the lower nobility under Franz von Sickingen
(1523), then the great Peasants' War, 1525. Both were defeated, chiefly in
consequence of the indecision of the parties most interested, the burghers of
the towns---an indecision into the causes of which we cannot here enter. From
that moment, the struggle degenerated into a fight between the local princes and
the central power, and ended by blotting out Germany, for 200 years, from the
politically active nations of Europe. The Lutheran Reformation produced a new
creed indeed, a religion adapted to absolute monarchy. No sooner were the
peasant of North-east Germany converted to Lutheranism than they were from
freemen reduced to serfs. 

But where Luther failed, Calvin won the day. Calvin's creed was one fit for the
boldest of the bourgeoisie of his time. His predestination doctrine was the
religious expression of the fact that in the commercial world of competition
success or failure does not depend upon a man's activity or cleverness, but upon
circumstances uncontrollable by him. It is not of him that willeth or of him
that runneth, but of the mercy of unknown superior economic powers; and this was
especially true at a period of economic revolution, when all old commercial
routes and centres were replaced by new ones, when India and America were opened
to the world, and when even the most sacred economic articles of faith---the
value of gold and silver---began to totter and to break down. Calvin's church
constitution of God was republicanized, could the kingdoms of this world remain
subject to monarchs, bishops, and lords? While German Lutheranism became a
willing tool in the hands of princes, Calvinism founded a republic in Holland,
and active republican parties in England, and, above all, Scotland.

In Calvinism, the second great bourgeois upheaval found its doctrine ready cut
and dried. This upheaval took place in England. The middle-class of the towns
brought it on, and the yeomanry of the country districts fought it out.
Curiously enough, in all the three great bourgeois risings, the peasantry
furnishes the army that has to do the fighting; and the peasantry is just the
class that, the victory once gained, is more surely ruined by the economic
consequences of that victory. A hundred years after Cromwell, the yeomanry of
England had almost disappeared. Anyhow, had it not been for that yeomanry and
for the \emph{plebian} element in the towns, the bourgeoisie alone would never
have fought the matter out to the bitter end, and would never have brought
Charles I to the scaffold. In order to secure even those conquests of the
bourgeoisie that were ripe for gathering at the time, the revolution had to be
carried considerably further---exactly as in 1793 in France and 1848 in Germany.
This seems, in fact, to be one of the laws of evolution of bourgeois society.

Well, upon this excess of revolutionary activity there necessarily followed the
inevitable reaction which, in its turn, went beyond the point where it might
have maintained itself. After a series of oscillations, the new centre of
gravity was at last attained and became a new starting-point. The grand period
of English history, known to respectability under the name of ``the Great
Rebellion'', and the struggles succeeding it, were brought to a close by the
comparatively puny events entitled by the Liberal historians ``the Glorious
Revolution''.

The new starting-point was a compromise between the rising middle-class and the
ex-feudal landowners. The latter, though called, as now, the aristocracy, had
been long since on the way which led them to become what Louis Philippe in
France became at a much later period: ``The first bourgeois of the kingdom''.
Fortunately for England, the old feudal barons had killed one another during the
War of the Roses. Their successors, though mostly scions of the old families,
had been so much out of the direct line of descent that they constituted quite a
new body, with habits and tendencies far more bourgeois than feudal. They fully
understood the value of money, and at once began to increase their rents by
turning hundreds of small farmers out and replacing them with sheep. Henry VIII,
while squandering the Church lands, created fresh bourgeois landlords by
wholesale; the innumerable confiscation of estates, regranted to absolute or
relative upstarts, and continued during the whole of the 17th century, had the
same result. Consequently, ever since Henry VII, the English ``aristocracy'',
far from counteracting the development of industrial production, had, on the
contrary, sought to indirectly profit thereby; and there had always been a
section of the great landowners willing, from economical or political reasons,
to cooperate with the leading men of the financial and industrial bourgeoisie.
The compromise of 1689 was, therefore, easily accomplished. The political spoils
of ``pelf and place'' were left to the great landowning families, provided the
economic interests of the financial, manufacturing, and commercial middle-class
were sufficiently attended to. And these economic interests were at that time
powerful enough to determine the general policy of the nation. There might be
squabbles about matters of detail, but, on the whole, the aristocratic oligarchy
knew too well that its own economic prosperity was irretrievably bound up with
that of the industrial and commercial middle-class.

From that time, the bourgeoisie was a humble, but still a recognized, component
of the ruling classes of England. With the rest of them, it had a common
interest in keeping in subjection the great working mass of the nation. The
merchant or manufacturer himself stood in the position of master, or, as it was
until lately called, of "natural superior" to his clerks, his work-people, his
domestic servants. His interest was to get as much and as good work out of them
as he could; for this end, they had to be trained to proper submission. He was
himself religious; his religion had supplied the standard under which he had
fought the king and the lords; he was not long in discovering the opportunities
this same religion offered him for working upon the minds of his natural
inferiors, and making them submissive to the behests of the masters it had
pleased God to place over them. In short, the English bourgeoisie now had to
take a part in keeping down the ``lower orders'', the great producing mass of
the nation, and one of the means employed for that purpose was the influence of
religion. 

There was another factor that contributed to strengthen the religious leanings
of the bourgeoisie. That was the rise of materialism in England. This new
doctrine not only shocked the pious feelings of the middle-class; it announced
itself as a philosophy only fit for scholars and cultivated men of the world,
in contrast to religion, which was good enough for the uneducated masses,
including the bourgeoisie. With Hobbes, it stepped on the stage as a defender of
royal prerogative and omnipotence; it called upon absolute monarchy to keep down
that \emph{puer robustus sed malitiosus}\footnote{robust but malicious boy}---to
wit, the people. Similarly, with the successors of Hobbes, with Bolingbroke,
Shaftesbury, etc., the new deistic form of materialism remained an aristocratic,
esoteric doctrine, and, therefore, hateful to the middle-class both for its
religious heresy and for its anti-bourgeois political connections. Accordingly,
in opposition to the materialism and deism of the aristocracy, those
Protestant sects which had furnished the flag and the fighting contingent
against the Stuarts continued to furnish the main strength of the progressive
middle-class, and form even today the backbone of ``the Great Liberal Party''.

In the meantime, materialism passed from England to France, where it met and
coalesced with another materialistic school of philosophers, a branch of
Cartesianism. In France, too, it remained at first an exclusively aristocratic
doctrine. But, soon, its revolutionary character asserted itself. The French
materialists did not limit their criticism to matters of religious belief; they
extended it to whatever scientific tradition or political institution they met
with; and to prove the claim of their doctrine to universal application, they
took the shortest cut, and boldly applied it to all subjects of knowledge in the
giant work after which they were named---the Encyclopedia. Thus, in one or the
other of its two forms---avowed materialism or deism---it became the creed of
the whole cultures youth of France; so much so that, when the Great Revolution
broke out, the doctrine hatched by English Royalists gave a theoretical flag to
French Republicans and Terrorists, and furnished the text for the Declaration of
the Rights of Man. The Great French Revolution was the third uprising of the
bourgeoisie, but the first that had entirely cast off the religious cloak, and
was fought out on undisguised political lines; it was the first, too, that was
really fought out up to the destruction of one of the combatants, the
aristocracy, and the complete triumpth of the other, the bourgeoisie. In
England, the continuity of pre-revolutionary and post-revolutionary
institutions, and the compromise between landlords and capitalists, found its
expression in the continuity of judicial precedents and in the religious
preservation of the feudal forms of the law. In France, the Revolution
constituted a complete breach with the traditions of the past; it cleared out
the very last vestiges of feudalism, and created in the Code Civil a masterly
adaptation of the old Roman law---that almost perfect expression of the
juridical relations corresponding to the economic stage called by Marx the
production of commodities---to modern capitalist conditions; so masterly that
this French revolutionary code still serves as a model for reforms of the law of
property in all other countries, not excepting England. Let us, however, not
forget that if English law continues to express the economic relations of
capitalist society in that barbarous feudal language which corresponds to the
thing expressed, just as English spelling corresponds to English
pronunciation---\emph{vous ecrivez Londres et vous prononcez Constantinople},
said a Frenchman---that same English law is the only one which has preserved
through ages, and transmitted to America and the Colonies, the best part of that
old Germanic personal freedom, local self-government, and independence from all
interference (but that of the law courts), which on the Continent has been lost
during the period of absolute monarchy, and has nowhere been as yet fully
recovered.

To return to our British bourgeois. The French Revolution gave him a splendid
opportunity, with the help of the Continental monarchies, to destroy French
maritime commerce, to annex French colonies, and to crush the last French
pretensions to maritime rivalry. That was one reason why he fought it. Another
was that the ways of this revolution went very much against his grain. Not only
its "execrable" terrorism, but the very attempt to carry bourgeois rule to
extremes. What should the British bourgeois do without his aristocracy, that
taught him hammers, such as they were, and invented fashions for him---that
furnished officers for the army, which kept order at home, and the navy, which
conquered colonial possessions and new markets abroad? There was, indeed, a
progressive minority of the bourgeoisie, that minority whose interests were not
so well attended to under the compromise; this section, composed chiefly of the
less wealthy middle-class, did sympathize with the Revolution, but it was
powerless in Parliament.

Thus, if materialism became the creed of the French revolution, the God-fearing
English bourgeois held all the faster to his religion. Had not the reign of
terror in Paris proved what was the upshot, if the religious instincts of the
masses were lost? The more materialism spread from France to neighbouring
countries, and was reinforced by similar doctrinal currents, notably by German
philosophy, the more, in fact, materialism and free thought generally became, on
the Continent, the necessary qualifications of a cultivated man, the more
stubbornly the English middle-class stuck to its manifold religious creeds.
These creeds might differ from one another, but they were, all of them,
distinctly religious, Christian creeds.

While the Revolution ensured the political triumph of the bourgeois in France,
in England Watt, Artwright, Cartwright, and others, initiated an industrial
revolution, which completely shifted the centre of gravity of economic power.
The wealth of the bourgeois increased considerably faster than that of the
landed aristocracy. Within the bourgeoisie itself, the financial aristocracy,
the bankers, etc., were more and more pushed into the background by the
manufacturers. The compromise of 1689, even after the gradual changes it had
undergone in favour of the bourgeoisie, no longer corresponded to the relative
position of the parties to it. The character of these parties, too, had changed;
the bourgeoisie of 1830 was very different from that of the preceding century.
The political power still left to the aristocracy, and used by them to resist
the pretensions of the new industrial bourgeoisie, became incompatible with the
new economic interests. A fresh struggle with the aristocracy was necessary; it
could end only in a victory of the new economic power. First, the Reform Act was
pushed through, in spite of all resistance, under the impulse of the French
Revolution of 1830. It gave to the bourgeoisie a recognized and powerful place
in Parliament. Then the Repeal of the Corn Laws (a move toward free-trade),
which settled, once and for all, the supremacy of the bourgeoisie, and
especially of its most active portion, the manufacturers, over the landed
aristocracy. This was the greatest victory of the bourgeoisie; it was, however,
also the last it gained in its own exclusive interest. Whatever triumphs it
obtained later on, it had to share with the new social power---first its ally,
but soon its rival.

The industrial revolution had created a class of large manufacturing
capitalists, but also a class---and a far more numerous one---of manufacturing
work-people. This class gradually increased in numbers, in proportion as the
industrial revolution seized upon one branch of manufacture after another, and
in the same proportion it increased its power. This power it proved as early as
1824, by forcing a reluctant Parliament to repeal the acts forbidding
combinations of workmen. During the Reform agitation, the workingmen constituted
the Radical wing of the Reform party; the Act of 1832 having excluded them from
the suffrage, they formulated their demands in the People's Charter, and
constituted themselves, in opposition to the great bourgeois Anti-Corn Law
party, into an independent party, the Chartists, the first working-men's party
of modern times.

Then came the Continental revolutions of February and March 1848, in which the
working people played such a prominent part, and, at least in Paris, put
forward demands which were certainly inadmissible from the point of view of
capitalist society. And then came the general reaction. First, the defeat of the
Chartists on April 10, 1848; then the crushing of the Paris workingmen's
insurrection in June of the same year; then the disasters of 1849 in Italy,
Hungary, South Germany, and at last the victory of Louis Bonaparte over Paris,
December 2, 1851. For a time, at least, the bugbear of working-class pretensions
was put down, but at what cost! If the British bourgeois had been convinced
before of the necessity of maintaining the common people in a religious mood,
how much more must he feel that necessity after all these experiences?
Regardless of the sneers of his Continental compeers, he continued to spend
thousands and tens of thousands, year after year, upon the evangelization of the
lower orders; not content with his own native religious machinery, he appealed
to Brother Jonathan\footnote{A sort of Anglo-Christian ``Uncle Sam''.}, the
greatest organizer in existence of religion as a trade, and imported from
America revivalism, Moody and Sankey, and the like; and, finally, he accepted
the dangerous aid of the Salvation Army, which revives the propaganda of early
Christianity, appeals to the poor as the elect, fights capitalism in a religious
way, and thus fosters an element of early Christian class antagonism, which one
day may become troublesome to the well-to-do people who now find the ready money
for it.

It seems a law of historical development that the bourgeoisie can in no European
country get hold of political power---at least for any length of time---in the
same exclusive way in which the feudal aristocracy kept hold of it during the
Middle Ages. Even in France, where feudalism was completely extinguished, the
bourgeoisie as a whole had held full possession of the Government for very short
periods only. During Louis Philippe's reign, 1830--48, a very small portion of
the bourgeoisie ruled the kingdom; by far the larger part were excluded from the
suffrage by the high qualification the Second Republic, 1848--51, the whole
bourgeoisie ruled but for three years only; their incapacity brought on the
Second Empire. It is only now, in the Third Republic, that the bourgeoisie as a
whole have kept possession of the helm for more than 20 years; and they are
already showing lively signs of decadence. A durable reign of the bourgeoisie
has been possible only in countries like America, where feudalism was unknown,
and society at the very beginning started from a bourgeois basis. And even in
France and America, the successors of the bourgeoisie, the working people, are
already knocking at the door.

In England, the bourgeoisie never held undivided sway. Even the victory of 1832
left the landed aristocracy in almost exclusive possession of all the leading
Government offices. The meekness with which the middle-class submitted to this
remained inconceivable to me until the great Liberal manufacturer, Mr. W. A.
Forster, in a public speech, implored the young men of Bradford to learn French,
as a mean to get on in the world, and quoted form his own experience how
sheepish he looked when, as a Cabinet Minister, he had to move in society where
French was, at least, as necessary as English! The fact was, the English
middle-class of that time were, as a rule, quite uneducated upstarts, and could
not help leaving to the aristocracy those superior Government places where other
qualifications were required than mere insular narrowness and insular conceit,
seasoned by business sharpness\endnote{
  And even in business matters, the conceit of national Chauvinism is but a
  sorry adviser. Up to quite recently, the average English manufacturer
  considered it derogatory for an Englishman to speak any language but his own,
  and felt rather proud than otherwise of the fact that ``poor devils'' of
  foreigners settled in England and took off his hands the trouble of disposing
  of his products abroad. He never noticed that these foreigners, mostly
  Germans, thus got command of a very large part of British foreign trade,
  imports and exports, and that the direct foreign trade of Englishmen became
  limited, almost entirely, to the colonies, China, the United States, and South
  America. Nor did he notice that these Germans traded with other Germans
  abroad, who gradually organized a complete network of commercial colonies all
  over the world. But, when Germany, about 40 years ago (c. 1850), seriously
  began manufacturing for export, this network served her admirably in her
  transformation, in so short a time, from a corn-exporting into a first-rate
  manufacturing country. Then, about 10 years ago, the British manufacturer got
  frightened, and asked his ambassadors and consuls how it was that he could no
  longer keep his customers together. The unanimous answer was: 
  \begin{enumerate}
    \item{You don't learn customer's language but expect him to speak your own;}
    \item{
      You don't even try to suit your customer's wants, habits, and tastes, but
      expect him to conform to your English ones.
    } 
  \end{enumerate}
}. Even now the endless newspaper debates about middle-class education show that
the English middle-class does not yet consider itself good enough for the best
education, and looks to something more modest. Thus, even after the repeal o
the Corn Laws, it appeared a matter of course that the men who had carried the
day---the Cobdens, Brights, Forsters, etc.---should remain excluded from a share
in the official government of the country, until 20 years afterwards a new
Reform Act opened to them the door of the Cabinet. The English bourgeois are, up
to the present day, so deeply penetrated by a sense of their social inferiority
that they keep up, at their own expense and that of the nation, an ornamental
caste of drones to represent that nation worthily at all State functions; and
they consider themselves highly honoured whenever one of themselves is found
worthy of admission into this select and privileged body, manufactured, after
all, by themselves.
% 6 paragraphs left

The industrial and commercial middle-class had, therefore, not yet succeeded in
driving the landed aristocracy completely from political power when another
competitor, the working-class, appeared on the stage. The reaction after the
Chartist movement and the Continental revolutions, as well as the unparalleled
extension of English trade from 1848--66 (ascribed vaguely to Free Trade alone,
but due far more to the colossal development of railways, ocean steamers, and
means of intercourse generally), had again driven the working-class into the
dependency of the Liberal party, of which they formed, as in pre-Chartist times,
the Radical wing. Their claims to the franchise, however, gradually became
irresistible; while the Whig leaders of the Liberals ``funked'', Disraeli showed
his superiority by making the Tories seize the favourable moment and introduce
household suffrage in the boroughs, along with a redistribution of seats. Then
followed the ballot; then, in 1884, the extension of household suffrage to the
countries and a fresh redistribution of seats, by which electoral districts
were, to some extent, equalized. All these measures considerable increased the
electoral power of the working-class, so much so that in at least 150 to 200
constituencies that class now furnished the majority of the voters. But
parliamentary government is a capital school for teaching respect for tradition;
if the middle-class look with awe and veneration upon what Lord John Manners
playfully called ``our old nobility'', the mass of the working-people then
looked up with respect and deference to what used to be designated as ``their
betters'', the middle-class. Indeed, the British workman, some 15 years ago, was
the model workman, whose respectful regard for the position of his master, and
whose self-restraining modesty in claiming rights for himself, consoled our
German economists of the \emph{Katheder-Socialist} school for the incurable
communistic and revolutionary tendencies of their own working-men at home.

But the English middle-class---good men of business as they are---saw farther
than the German professors. They had shared their powers but reluctantly with
the working-class. They had learnt, since that time, they had been compelled to
incorporate the better part of the People's Charter on the Statues of the United
Kingdom. Now, if ever, the people must be kept in order by moral means, and the
first and foremost of all moral means of action upon the masses is and
remains---religion. Hence, the parsons' majorities on the School Boards, hence
the increasing self-taxation of the bourgeoisie for the support of all sorts of
revivalism, from ritualism to the Salvation Army.

And now came the triumph of British respectability over the free thought and
religious laxity of the Continental bourgeois.

\printendnotes
