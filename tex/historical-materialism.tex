\chapter{Historical Materialism}

The materialist conception of history starts from the proposition that the
production of the means to support human life and, next to production, the
exchange of things produced, is the basis of all social structure; that in every
society that has appeared in history, the manner in which wealth is distributed
and society divided into classes or orders is dependent upon what is produced,
how it is produced, and how the products are exchanged. From this point of
view, the final causes of all social changes and political revolutions are to be
sought, not in men's brains, not in men's better insights into eternal truth and
justice, but in changes in the modes of production and exchange. They are to be
sought, not in the \emph{philosophy}, but in the \emph{economics} of each
particular epoch. The growing perception that existing social institutions are
unreasonable and unjust, that reason has become unreason, and right
wrong\footnote{Mephistopheles in Goethe's \emph{Faust}.}, is only proof that in
the modes of production and exchange changes have silently taken place with
which the social order, adapted to earlier economic conditions, is no longer in
keeping. From this it also follows that the means of getting rid of the
incongruities that have been brought to light must also be present, in a more
or less developed condition, within the changed modes of production themselves.
These means are not to be invented by deduction from fundamental principles,
but are to be discovered in the stubborn facts of the existing system of
production.

What is, then, the position of modern Socialism in this connection?

The present situation of society---this is now pretty generally conceded---is
the creation of the ruling class of today, of the bourgeoisie. The mode of
production peculiar to the bourgeoisie, known, since Marx, as the capitalist
mode of production, was incompatible with the feudal system, with the privileges
it conferred upon individuals, entire social ranks and local corporations, as
well as with the hereditary ties of subordination which constituted the
framework of its social organization. The bourgeoisie broke up the feudal system
and built upon its ruins the capitalist order of society, the kingdom of free
competition, of personal liberty, of the equality, before the law, of all
commodity owners, of all the rest of the capitalist blessings. Thenceforward,
the capitalist mode of production could develop in freedom. Since steam,
machinery, and the making of machines by machinery transformed the older
manufacture into modern industry, the productive forces, evolved under the
guidance of the bourgeoisie, developed with a rapidity and in a degree unheard
of before. But just as the older manufacture, in its time, and handicraft,
becoming more developed under its influence, had come into collision with the
feudal trammels of the guilds, so now modern industry, in its complete
development, comes into collision with the bounds within which the capitalist
mode of production holds it confined. The new productive forces have already
outgrown the capitalistic mode of using them. And this conflict between
productive forces and modes of production is not a conflict engendered in the
mind of man, like that between original sin and divine justice. It exists, in
fact, objectively, outside us, independently of the will and actions even of the
men that have brought it on. Modern Socialism is nothing but the reflex, in
thought, of this conflict in fact; its ideal reflection in the minds, first, of
the class directly suffering under it, the working class.

Now, in what does this conflict exist?

Before capitalist production---i.e., in the Middle Ages---the system of petty
industry obtained generally, based upon the private property of the labourers in
their means of production; in the country, the agriculture of the small peasant,
freeman, or serf; in the towns, the handicrafts organized in guilds. The
instruments of labour of single individuals, adapted for the use of one worker,
and, therefore, of necessity, small, dwarfish, circumscribed. But, for this very
reason, they belonged as a rule to the producer himself. To concentrate these
scattered, limited means of production, to enlarge them, to turn them into the
powerful levels of production of the present day---this was precisely the
historic role of capitalist production and of its upholder, the bourgeoisie. In
the fourth section of \emph{Capital}, Marx has explained in detail how since the
15th century this has been historically worked out through the three phases of
simple co-operation, manufacture, and modern industry. But the bourgeoisie, as
is shown there, could not transform these puny means of production into mighty
productive forces without transforming them, at the same time, from means of
production of the individual into \emph{social} means of production only
workable by a collectivity of men. The spinning wheel, the handloom, the
blacksmith's hammer, were replaced by the spinning-machine, the power-loom, the
steam-hammer; the individual workshop, by the factory implying the co-operation
of hundreds and thousands of workmen. In like manner, production itself changed
from a series of individual into a series of social acts, and the production
from individual to social products. The yarn, the cloth, the metal articles that
now come out of the factory were the joint product of many workers, through
whose hands they had successively to pass before they were ready. No one person
could say of them: ``I made that; this is \emph{my} product''.

But where, in a given society, the fundamental form of production is that
spontaneous division of labour which creeps in gradually and not upon any
preconceived plan, there the products take on the form of \emph{commodities},
whose mutual exchange, buying and selling, enable the individual producers to
satisfy their manifold wants. And this was the case in the Middle Ages. The
peasant, for example, sold to the artisan agricultural products and bought from
him the products of handicraft. Into this society of individual producers, of
commodity producers, the new mode of production thrust itself. In the midst of
the old division of labour, grown up spontaneously and upon \emph{no definite
plan}, which had governed the whole of society, now arose division of labour
upon \emph{a definite plan}, as organized in the factory; side by side with
\emph{individual} production appeared \emph{social} production. The products of
both were sold in the same market, and, therefore, at prices at least
approximately equal. But organization upon a definite plan was stronger than
spontaneous division of labour. The factories working with the combined social
forces of a collectivity of individuals produced their commodities far more
cheaply than the individual small producers. Individual producers succumbed
in one department after another. Socialized production revolutionized all the
old methods of production. But its revolutionary character was, at the same
time, so little recognized that it was, on the contrary, introduced as a means
of increasing and developing the production of commodities. When it arose, it
found ready-made, and made liberal use of, certain machinery for the production
and exchange of commodities: merchants' capital, handicraft, wage-labour.
Socialized production thus introducing itself as a new form of the production
of commodities, it was a matter of course that under it the old forms of
appropriation remained in full swing, and were applied to its products as well.

In the medieval stage of evolution of the production of commodities, the
question as to the owner of the product of labour could not arise. The
individual producer, as a rule, had, from raw material belonging to himself, and
generally his own handiwork, produced it with his own tools, by the labour of
his own hands or of his family. There was no need for him to appropriate the new
product. It belonged wholly to him, as a matter of course. His property in the
product was, therefore, based \emph{upon his own labour}. Even where external
help was used, this was, as a rule, of little importance, and very generally was
compensated by something other than wages. The apprentices and journeymen of the
guilds worked less for board and wages than for education, in order that they
might become master craftsmen themselves.

Then came the concentration of the means of production and of the producers in
large workshops and manufactories, their transformation into actual socialized
means of production and socialized producers. But the socialized producers and
means of production and their products were still treated, after this change,
just as they had been before---i.e., as the means of production and the products
of individuals. Hitherto, the owner of the instruments of labour had himself
appropriated the product, because, as a rule, it was his own product and the
assistance of others was the exception. Now, the owner of the instruments of
labour always appropriated to himself the product, although it was no longer
\emph{his} product but exclusively the product of the \emph{labour of others}.
Thus, the products now produced socially were not appropriated by those who had
actually set in motion the means of production and actually produced the
commodities, but by the \emph{capitalists}. The means of production, and
production itself, had become in essence socialized. But they were subjected to
a form of appropriation which presupposes the private production of individuals,
under which, therefore, everyone owns his own product and brings it to market.
The mode of production is subjected to this form of appropriation, although it
abolishes the conditions upon which the latter rests\endnote{
  It is hardly necessary in this connection to point out that, even if the form
  of appropriation remains the same, the character of the appropriation is just
  as much revolutionized as production is by the changes described above. It is,
  of course, a very different matter whether I appropriate to myself my own
  product or that of another. Note in passing that wage-labor, which contains
  the whole capitalist mode of production in embryo, is very ancient; in a
  sporadic, scattered form, it existed for centuries alongside slave-labor. But
  the embryo could duly develop into the capitalistic mode of production only
  when the necessary historical pre-conditions had been furnished.
}.

This contradiction, which gives to the new mode of production its capitalistic
character, \emph{contains the germ of the whole of the social antagonisms of
today}. The greater the mastery obtained by the new mode of production over all
important fields of production and in all manufacturing countries, the more it
reduced individual production to an insignificant residuum, \emph{the more
clearly was brought out the incompatibility of socialized production with
capitalistic appropriation}.
