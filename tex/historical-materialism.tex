\chapter{Historical Materialism}

The materialist conception of history starts from the proposition that the
production of the means to support human life and, next to production, the
exchange of things produced, is the basis of all social structure; that in every
society that has appeared in history, the manner in which wealth is distributed
and society divided into classes or orders is dependent upon what is produced,
how it is produced, and how the products are exchanged. From this point of
view, the final causes of all social changes and political revolutions are to
be sought, not in men's brains, not in men's better insights into eternal truth
and justice, but in changes in the modes of production and exchange. They are
to be sought, not in the \emph{philosophy}, but in the \emph{economics} of each
particular epoch. The growing perception that existing social institutions are
unreasonable and unjust, that reason has become unreason, and right
wrong\footnote{Mephistopheles in Goethe's \emph{Faust}.}, is only proof that in
the modes of production and exchange changes have silently taken place with
which the social order, adapted to earlier economic conditions, is no longer
in keeping. From this it also follows that the means of getting rid of the
incongruities that have been brought to light must also be present, in a more
or less developed condition, within the changed modes of production themselves.
These means are not to be invented by deduction from fundamental principles,
but are to be discovered in the stubborn facts of the existing system of
production.

What is, then, the position of modern Socialism in this connection?
