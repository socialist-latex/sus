\chapter{Historical Materialism}

The materialist conception of history starts from the proposition that the
production of the means to support human life and, next to production, the
exchange of things produced, is the basis of all social structure; that in every
society that has appeared in history, the manner in which wealth is distributed
and society divided into classes or orders is dependent upon what is produced,
how it is produced, and how the products are exchanged. From this point of
view, the final causes of all social changes and political revolutions are to be
sought, not in men's brains, not in men's better insights into eternal truth and
justice, but in changes in the modes of production and exchange. They are to be
sought, not in the \emph{philosophy}, but in the \emph{economics} of each
particular epoch. The growing perception that existing social institutions are
unreasonable and unjust, that reason has become unreason, and right
wrong\footnote{Mephistopheles in Goethe's \emph{Faust}.}, is only proof that in
the modes of production and exchange changes have silently taken place with
which the social order, adapted to earlier economic conditions, is no longer in
keeping. From this it also follows that the means of getting rid of the
incongruities that have been brought to light must also be present, in a more
or less developed condition, within the changed modes of production themselves.
These means are not to be invented by deduction from fundamental principles,
but are to be discovered in the stubborn facts of the existing system of
production.

What is, then, the position of modern Socialism in this connection?

The present situation of society---this is now pretty generally conceded---is
the creation of the ruling class of today, of the bourgeoisie. The mode of
production peculiar to the bourgeoisie, known, since Marx, as the capitalist
mode of production, was incompatible with the feudal system, with the privileges
it conferred upon individuals, entire social ranks and local corporations, as
well as with the hereditary ties of subordination which constituted the
framework of its social organization. The bourgeoisie broke up the feudal system
and built upon its ruins the capitalist order of society, the kingdom of free
competition, of personal liberty, of the equality, before the law, of all
commodity owners, of all the rest of the capitalist blessings. Thenceforward,
the capitalist mode of production could develop in freedom. Since steam,
machinery, and the making of machines by machinery transformed the older
manufacture into modern industry, the productive forces, evolved under the
guidance of the bourgeoisie, developed with a rapidity and in a degree unheard
of before. But just as the older manufacture, in its time, and handicraft,
becoming more developed under its influence, had come into collision with the
feudal trammels of the guilds, so now modern industry, in its complete
development, comes into collision with the bounds within which the capitalist
mode of production holds it confined. The new productive forces have already
outgrown the capitalistic mode of using them. And this conflict between
productive forces and modes of production is not a conflict engendered in the
mind of man, like that between original sin and divine justice. It exists, in
fact, objectively, outside us, independently of the will and actions even of the
men that have brought it on. Modern Socialism is nothing but the reflex, in
thought, of this conflict in fact; its ideal reflection in the minds, first, of
the class directly suffering under it, the working class.

Now, in what does this conflict exist?

Before capitalist production---i.e., in the Middle Ages---the system of petty
industry obtained generally, based upon the private property of the labourers in
their means of production; in the country, the agriculture of the small peasant,
freeman, or serf; in the towns, the handicrafts organized in guilds. The
instruments of labour of single individuals, adapted for the use of one worker,
and, therefore, of necessity, small, dwarfish, circumscribed. But, for this very
reason, they belonged as a rule to the producer himself. To concentrate these
scattered, limited means of production, to enlarge them, to turn them into the
powerful levels of production of the present day---this was precisely the
historic role of capitalist production and of its upholder, the bourgeoisie. In
the fourth section of \emph{Capital}, Marx has explained in detail how since the
15th century this has been historically worked out through the three phases of
simple cooperation, manufacture, and modern industry. But the bourgeoisie, as
is shown there, could not transform these puny means of production into mighty
productive forces without transforming them, at the same time, from means of
production of the individual into \emph{social} means of production only
workable by a collectivity of men. The spinning wheel, the handloom, the
blacksmith's hammer, were replaced by the spinning-machine, the power-loom, the
steam-hammer; the individual workshop, by the factory implying the cooperation
of hundreds and thousands of workmen. In like manner, production itself changed
from a series of individual into a series of social acts, and the production
from individual to social products. The yarn, the cloth, the metal articles that
now come out of the factory were the joint product of many workers, through
whose hands they had successively to pass before they were ready. No one person
could say of them: ``I made that; this is \emph{my} product''.

But where, in a given society, the fundamental form of production is that
spontaneous division of labour which creeps in gradually and not upon any
preconceived plan, there the products take on the form of \emph{commodities},
whose mutual exchange, buying and selling, enable the individual producers to
satisfy their manifold wants. And this was the case in the Middle Ages. The
peasant, for example, sold to the artisan agricultural products and bought from
him the products of handicraft. Into this society of individual producers, of
commodity producers, the new mode of production thrust itself. In the midst of
the old division of labour, grown up spontaneously and upon \emph{no definite
plan}, which had governed the whole of society, now arose division of labour
upon \emph{a definite plan}, as organized in the factory; side by side with
\emph{individual} production appeared \emph{social} production. The products of
both were sold in the same market, and, therefore, at prices at least
approximately equal. But organization upon a definite plan was stronger than
spontaneous division of labour. The factories working with the combined social
forces of a collectivity of individuals produced their commodities far more
cheaply than the individual small producers. Individual producers succumbed
in one department after another. Socialized production revolutionized all the
old methods of production. But its revolutionary character was, at the same
time, so little recognized that it was, on the contrary, introduced as a means
of increasing and developing the production of commodities. When it arose, it
found ready-made, and made liberal use of, certain machinery for the production
and exchange of commodities: merchants' capital, handicraft, wage-labour.
Socialized production thus introducing itself as a new form of the production
of commodities, it was a matter of course that under it the old forms of
appropriation remained in full swing, and were applied to its products as well.

In the medieval stage of evolution of the production of commodities, the
question as to the owner of the product of labour could not arise. The
individual producer, as a rule, had, from raw material belonging to himself, and
generally his own handiwork, produced it with his own tools, by the labour of
his own hands or of his family. There was no need for him to appropriate the new
product. It belonged wholly to him, as a matter of course. His property in the
product was, therefore, based \emph{upon his own labour}. Even where external
help was used, this was, as a rule, of little importance, and very generally was
compensated by something other than wages. The apprentices and journeymen of the
guilds worked less for board and wages than for education, in order that they
might become master craftsmen themselves.

Then came the concentration of the means of production and of the producers in
large workshops and manufactories, their transformation into actual socialized
means of production and socialized producers. But the socialized producers and
means of production and their products were still treated, after this change,
just as they had been before---i.e., as the means of production and the products
of individuals. Hitherto, the owner of the instruments of labour had himself
appropriated the product, because, as a rule, it was his own product and the
assistance of others was the exception. Now, the owner of the instruments of
labour always appropriated to himself the product, although it was no longer
\emph{his} product but exclusively the product of the \emph{labour of others}.
Thus, the products now produced socially were not appropriated by those who had
actually set in motion the means of production and actually produced the
commodities, but by the \emph{capitalists}. The means of production, and
production itself, had become in essence socialized. But they were subjected to
a form of appropriation which presupposes the private production of individuals,
under which, therefore, everyone owns his own product and brings it to market.
The mode of production is subjected to this form of appropriation, although it
abolishes the conditions upon which the latter rests\endnote{
  It is hardly necessary in this connection to point out that, even if the form
  of appropriation remains the same, the character of the appropriation is just
  as much revolutionized as production is by the changes described above. It is,
  of course, a very different matter whether I appropriate to myself my own
  product or that of another. Note in passing that wage-labour, which contains
  the whole capitalist mode of production in embryo, is very ancient; in a
  sporadic, scattered form, it existed for centuries alongside slave-labour. But
  the embryo could duly develop into the capitalistic mode of production only
  when the necessary historical pre-conditions had been furnished.
}.

This contradiction, which gives to the new mode of production its capitalistic
character, \emph{contains the germ of the whole of the social antagonisms of
today}. The greater the mastery obtained by the new mode of production over all
important fields of production and in all manufacturing countries, the more it
reduced individual production to an insignificant residuum, \emph{the more
clearly was brought out the incompatibility of socialized production with
capitalistic appropriation}.

We have seen that the capitalistic mode of production thrust its way into a
society of commodity-producers, of individual producers, whose social bond was
the exchange of their products. But every society based upon the production of
commodities has this peculiarity: that the producers have lost control over
their own social inter-relations. Each man produces for himself with such means
of production as he may happen to have, and for such exchange as he may require
to satisfy his remaining wants. No one knows how much of his particular article
is coming on the market, nor how much of it will be wanted. No one knows whether
his individual product will meet an actual demand, whether he will be able to
make good his costs of production or even to sell his commodity at all. Anarchy
reigns in socialized production.

But the production of commodities, like every other form of production, has it
peculiar, inherent laws inseparable from it; and these laws work, despite
anarchy in and through anarchy. They reveal themselves in the only persistent
form of social inter-relations---i.e., in exchange---and here they affect the
individual producers as compulsory laws of competition. They are, at first,
unknown to these producers themselves, and have to be discovered by them
gradually and as the result of experience. They work themselves out, therefore,
independently of the producers, and in antagonism to them, as inexorable natural
laws of their particular form of production. The product governs the producers.

In medieval society, especially in the earlier centuries, production was
essentially directed toward satisfying the wants of the individual. It
satisfied, in the main, only the wants of the producer and his family. Where
relations of personal dependence existed, as in the country, it also helped to
satisfy the wants of the feudal lord. In all this there was, therefore, no
exchange; the products, consequently, did not assume the character of
commodities. The family of the peasant produced almost everything they wanted:
clothes and furniture, as well as the means of subsistence. Only when it began
to produce more than was sufficient to supply its own wants and the payments in
kind to the feudal lords, only then did it also produce commodities. This
surplus, thrown into socialized exchange and offered for sale, became
commodities.

The artisan in the towns, it is true, had from the first to produce for
exchange. But they, also, themselves supplied the greatest part of their
individual wants. They had gardens and plots of land. They turned their cattle
out into the communal forest, which, also, yielded them timber and firing. The
women spun flax, wool, and so forth. Production for the purpose of exchange,
production of commodities, was only in its infancy. Hence, exchange was
restricted, the market narrow, the methods of production stable; there was local
exclusiveness without, local unity within; the mark in the country; in the town,
the guild.

But with the extension of the production of commodities, and especially with the
introduction of the capitalist mode of production, the laws of
commodity-production, hitherto latent, came into action more openly and with
greater force. The old bonds were loosened, the old exclusive limits broken
through, the producers were more and more turned into independent, isolated
producers of commodities. It became apparent that the production of society at
large was ruled by absence of plan, by accident, by anarchy; and this anarchy
grew to a greater and greater height. But the chief means by aid of which the
capitalist mode of production intensified this anarchy of socialized production
was the exact opposite of anarchy. It was the increasing organization of
production, upon a social basis, in every individual productive establishment.
By this, the old, peaceful, stable condition of things was ended. Whenever this
organization of production was introduced into a branch of industry, it brooked
no other method of production by its side. The field of labour became a
battleground. The great geographical discoveries, and the colonization following
them, multiplied markets and quickened the transformation of handicraft into
manufacture. The war did not simply break out between the individual producers
of particular localities. The local struggles begat, in their turn, national
conflicts, the commercial wars of the 17th and 18th centuries.

Finally, modern industry and the opening of the world market made the struggle
universal, and at the same time gave it an unheard-of virulence. Advantages in
natural or artificial conditions of production now decide the existence or
non-existence of individual capitalists, as well as of whole industries and
countries. He that falls is remorselessly cast aside. It is the Darwinian
struggle of the individual for existence transferred from Nature to society with
intensified violence. The conditions of existence natural to the animal appear
as the final term of human development. The contradiction between socialized
production and capitalistic appropriation now presents itself as \emph{an
antagonism between the organization of production in the individual workshop and
the anarchy of production in society generally}.

But the perfecting of machinery is making human labour superfluous. If the
introduction and increase of machinery means the displacement of millions of
manual by a few machine-workers, improvement in machinery means the displacement
of more and more of the machine-workers themselves. It means, in the last
instance, the production of a number of available wage workers in excess of the
average needs of capital, the formation of a complete industrial reserve army,
as I called it in 1845\footnote{
  From \emph{The Conditions of the Working-Class in England}, Sonnenschein \&
Co., p. 84.
}, available at the times when industry is working at high pressure, to be cast
out upon the street when the inevitable crash comes, a constant dead weight upon
the limbs of the working-class in its struggle for existence with capital, a
regulator for keeping of wages down to the low level that suits the interests of
capital.

Thus it comes about, to quote Marx, that machinery becomes the most powerful
weapon in the war of capital against the working-class; that the instruments of
labour constantly tear the means of subsistence out of the hands of the
labourer; that the very product of the worker is turned into an instrument for
his subjugation.

Thus it come about that the economizing of the instruments of labour becomes at
the same time, from the outset, the most reckless waste of labour-power, and
robbery based upon the normal conditions under which labour functions; that
machinery,
%
\begin{quote}
  ``the most powerful instrument for shortening labour time, becomes the most
  unfailing means for placing every moment of the labourer's time, and that of
  his family at the disposal of the capitalist for the purpose of expanding the
  value of his capital''. [\emph{Capital}, p. 406]
\end{quote}

Thus it comes about that the overwork of some becomes the preliminary condition
for the idleness of others, and that modern industry, which hunts after new
consumers over the whole world, forces the consumption of the masses at home
down to a starvation minimum, and in doing thus destroys its own home market.
%
\begin{quote}
  ``The law that always equilibriates the relative surplus- population, or
  industrial army, to the extent and energy of accumulation, this law rivets the
  labourer to capital more firmly than the wedges of Vulcan did Prometheus to
  the rock. It establishes an accumulation of misery, corresponding with the
  accumulation of capital. Accumulation of wealth at one pole is, therefore, at
  the same time accumulation of misery, agony of toil, slavery, ignorance,
  brutality, mental degradation, at the opposite pole, i.e., on the side of the
  class that produces \emph{its own product in the form of capital}.''
  [\emph{Capital}, p. 661]
\end{quote}
%
And to expect any other division of the products from the capitalist mode of
production is the same as expecting the electrodes of a battery not to decompose
acidulated water, not to liberate oxygen at the positive, hydrogen at the
negative pole, so long as they are connected with the battery.

We have seen that the ever-increasing perfectibility of modern machinery is, by
the anarchy of social production, turned into a compulsory law that forces the
individual industrial capitalist always to improve his machinery, always to
increase its productive force. The bare possibility of extending the field of
production is transformed for him into a similarly compulsory law. The enormous
expansive force of modern industry, compared with which that of gases is mere
child's play, appears to us now as a \emph{necessity} for expansion, both
qualitative and quantitative, that laughs at all resistance. Such resistance is
offered by consumption, by sales, by the markets for the products of modern
industry. But the capacity for extension, extensive and intensive, of the
markets is primarily governed by quite different laws that work much less
energetically. The extension of the markets cannot keep pace with the extension
of production. The collision becomes inevitable, and as this cannot produce any
real solution so long as it does not break in pieces the capitalist mode of
production, the collisions become periodic. Capitalist production has begotten
another ``vicious circle''.

As a matter of fact, since 1825, when the first general crisis broke out, the
whole industrial and commercial world, production and exchange among all
civilized peoples and their more or less barbaric hangers-on, are thrown out of
joint about once every 10 years. Commerce is at a stand-still, the markets are
glutted, products accumulate, as multitudinous as they are unsaleable, hard cash
disappears, credit vanishes, factories are closed, the mass of the workers are
in want of the means of subsistence, because they have produced too much of the
means of subsistence; bankruptcy follows upon bankruptcy, execution upon
execution. The stagnation lasts for years; productive forces and products are
wasted and destroyed wholesale, until the accumulated mass of commodities
finally filter off, more or less depreciated in value, until production and
exchange gradually begin to move again. Little by little, the pace quickens. It
becomes a trot. The industrial trot breaks into a canter, the canter in turn
grows into the headlong gallop of a perfect steeplechase of industry, commercial
credit, and speculation, which finally, after breakneck leaps, ends where it
began---in the ditch of a crisis. And so over and over again. We have now, since
the year 1825, gone through this five times, and at the present moment (1877),
we are going through it for the sixth time. And the character of these crises is
so clearly defined that Fourier hit all of them off when he described the first
``crisis plethorique'', a crisis from plethora.

In these crises, the contradiction between socialized production and capitalist
appropriation ends in a violence explosion. The circulation of commodities is,
for the time being, stopped. Money, the means of circulation, becomes a
hindrance to circulation. All the laws of production and circulation of
commodities are turned upside down. The economic collision has reached its
apogee. \emph{The mode of production is in rebellion against the mode of
exchange.}

The fact that the socialized organization of production within the factory has
developed so far that it has become incompatible with the anarchy of production
in society, which exists side by side with and dominates it, is brought home to
the capitalist themselves by the violent concentration of capital that occurs
during crises, through the ruin of many large, and a still greater number of
small, capitalists. The whole mechanism of the capitalist mode of production
breaks down under the pressure of the productive forces, its own creations. It
is no longer able to turn all this mass of means of production into capital.
They lie fallow, and for that very reason the industrial reserve army must also
lie fallow. Means of production, means of subsistence, available labourers, all
the elements of production and of general wealth, are present in abundance. But
``abundance becomes the source of distress and want'' (Fourier), because it is
the very thing that prevents the transformation of the means of production and
subsistence into capital. For in a capitalistic society, the means of production
and subsistence into capital. For in capitalistic society, the means of
production can only function when they have undergone a preliminary
transformation into capital, into the means of exploiting human labour-power.
The necessity of this transformation into capital of the means of production and
subsistence stands like a ghost between these and the workers. It alone prevents
the coming together of the material and personal levers of production; it alone
forbids the means of production to function, the workers to work and live. On
the one hand, therefore, the capitalistic mode of production stands convicted of
its own incapacity to further direct these productive forces. On the other,
these productive forces themselves, with increasing energy, press forward to the
removal of the existing contradiction, to the abolition of their quality as
capital, to the \emph{practical recognitions of their character as social
production forces}.

This rebellion of the productive forces, as they grow more and more powerful,
against their quality as capital, this stronger and stronger command that their
social character shall be recognized, forces the capital class itself to treat
them more and more as social productive forces, so far as this is possible under
capitalist conditions. The period of industrial high pressure, with its
unbounded inflation of credit, not less than the crash itself, by the collapse
of great capitalist establishments, tends to bring about that form of the
socialization of great masses of the means of production which we meet with in
the different kinds of joint-stock companies. Many of these means of production
and of distribution are, from the outset, so colossal that, like the railways,
they exclude all the other forms of capitalistic expansion. At a further stage
of evolution, this form also becomes insufficient. The producers on a larger
scale in a particular branch of an industry in a particular country unite in a
``Trust'', a union for the purpose of regulating production. They determine the
total amount to be produced, parcel it out among themselves, and thus enforce
the selling price fixed beforehand. But trusts of this kind, as soon as business
becomes bad, are generally liable to break up, and on this very account compel a
yet greater concentration of association. The internal competition gives place
to the internal monopoly of this one company. This has happened in 1890 with the
English alkali production, which is now, after the fusion of 48 large works, in
the hands of one company, conducted upon a single plan, and with a capital of
£6\,000\,000.

In the trusts, freedom of competition changes into its very opposite---into
monopoly; and the production without any definite plan of capitalistic society
capitulates to the production upon a definite plan of the invading socialistic
society. Certainly, this is so far still to the benefit and advantage of the
capitalists. But, in this case, the exploitation is so palpable, that it must
break down. No nation will put up with production conducted by trusts, with so
barefaced an exploitation of the community by a small band of dividend-mongers. 

In any case, with trusts or without, the official representative of capitalist
society---the state---will ultimately have to undertake the direction of
production\endnote{
  I say ``have to''. For only when the means of production and distribution have
  \emph{actually} outgrown the form of management by joint-stock companies, and
  when, therefore, the taking them over by the State has become
  \emph{economically} inevitable, only then---even if it is the State of today
  that effects this---is there an economic advance, the attainment of another
  step preliminary to the taking over of all productive forces by society
  itself. But of late, since Bismarck went in for State-ownership of industrial
  establishments, a kind of spurious Socialism has arisen, degenerating, now and
  again, into something of flunkyism, that without more ado declares \emph{all}
  State-ownership, even of the Bismarkian sort, to be socialistic. Certainly, if
  the taking over by the state of the tobacco industry is socialistic, then
  Napoleon and Metternich must be numbered among the founders of Socialism.

  If the Belgian State, for quite ordinary political and financial reasons,
  itself constructed its chief railway lines; if Bismarck, not under any
  economic compulsion, took over for the State the chief Prussian lines, simply
  to be the better able to have them in hand in case of war, to bring up the
  railway employees as voting cattle for the Government, and especially to
  create for himself a new source of income independent of parliamentary
  votes---this was, in no sense, a socialistic measure, directly or indirectly,
  consciously or unconsciously. Otherwise, the Royal Maritime Company, the Royal
  porcelain manufacture, and even the regimental tailor of the army would also
  be socialistic institutions, or even, as was seriously proposed by a sly dog
  in Frederick William III's reign, the taking over by the State of the
  brothels.
}. This necessity for conversion into State property is felt first in the great
institutions for intercourse and communication---the post office, the
telegraphs, the railways.

If the crises demonstrate the incapacity of the bourgeoisie for managing any
longer modern productive forces, the transformation of the great establishments
for production and distribution into joint-stock companies, trusts, and State
property, show how unnecessary the bourgeoisie are for that purpose. All the
social functions of the capitalist has no further social function than that of
pocketing dividends, tearing off coupons, and gambling on the Stock Exchange,
where the different capitalists despoil one another of their capital. At first,
the capitalistic mode of production forces out the workers. Now, it forces out
the capitalists, and reduces them, just as it reduced the workers, to the ranks
of the surplus-population, although not immediately into those of the industrial
reserve army. 

But, the transformation---either into joint-stock companies and trusts, or into
State-ownership---does not do away with the capitalistic nature of the
productive forces. In the joint-stock companies and trusts, this is obvious. And
the modern State, again, is only the organization that bourgeois society takes
on in order to support the external conditions of the capitalist mode of
production against the encroachments as well of the workers as of individual
capitalists. The modern state, no matter what its form, is essentially a
capitalist machine---the state of the capitalists, the ideal personification of
the total national capital. The more it proceeds to the taking over of
productive forces, the more does it actually become the national capitalist, the
more citizens does it exploit. The workers remain wage-workers---proletarians.
The capitalist relation is not done away with. It is rather, brought to a head.
But, brought to a head, it topples over. State-ownership of the productive
forces is not the solution of this conflict, but concealed within it are the
technical conditions that form the elements of that solution.

This solution can only consist in the practical recognition of the social nature
of the modern forces of production, and therefore in the harmonizing with the
socialized character of the means of production. And this can only come about
by society openly and directly taking possession of the productive forces which
have outgrown all control, except that of society as a whole. The social
character of the means of production and of the products today reacts against
the producers, periodically disrupts all production and exchange, acts only
like a law of Nature working blindly, forcibly, destructively. But, with the
taking over by society of the productive forces, the social character of the
means of production and of the products will be utilized by the producers with a
perfect understanding of its nature, and instead of being a source of
disturbance and periodical collapse, will become the most powerful lever of
production itself. 

Active social forces work exactly like natural forces: blindly, forcibly,
destructively, so long as we do not understand, and reckon with, them. But, when
once we understand them, when once we grasp their action, their direction, their
effects, it depends only upon ourselves to subject them more and more to our own
will, and, by means of them, to reach our own ends. And this holds quite
especially of the mighty productive forces of today. As long as we obstinately
refuse to understand the nature and the character of these social means of
action---and this understanding goes against the grain of the capitalist mode of
production, and its defenders---so long these forces are at work in spite of us,
in opposition to us, so long they master us, as we have shown above in detail.

But when once their nature is understood, they can, in the hand working
together, be transformed from master demons into willing servants. The
difference is as that between the destructive force of electricity in the
lightning in the storm, and electricity under command in the telegraph and the
voltaic arc; the difference between a conflagration, and fire working in the
service of man. With this recognition, at last, of the real nature of the
productive forces of today, the social anarchy of production gives place to a
social regulation of production upon a definite plan, according to the needs of
the community and of each individual. Then the capitalist mode of appropriation,
in which the product enslaves first the producer, and then the appropriator, is
replaced by the mode of appropriation of the products that is based upon the
nature of the modern means of production; upon the one hand, direct social
appropriation, as means to the maintenance and extension of production---on the
other, direct individual appropriation, as means of subsistence and of
enjoyment. 

Whilst the capitalist mode of production more and more completely transforms the
great majority of the population into proletarians, it creates the power which,
under penalty of its own destruction, is forced to accomplish this revolution.
Whilst it forces on more and more of the transformation of the vast means of
production, already socialized, into State property, it shows itself the way to
accomplishing this revolution. \emph{The proletariat seizes political power and
turns the means of production into State property.}

But, in doing this, it abolishes itself as proletariat, abolishes all class
distinction and class antagonisms, abolishes also the State as State. Society,
thus far, based upon class antagonisms, had need of the State. That is, of an
organization of the particular class which was, pro tempore, the exploiting
class, an organization for the purpose of preventing any interference from
without with the existing conditions of production, and, therefore, especially,
for the purpose of forcibly keeping the exploited classes in the condition of
oppression corresponding with the given more of production (slavery, serfdom,
wage-labour). The State was the official representative of society as a whole;
the gathering of it together into a visible embodiment. But, it was this only in
so far as it was the State of that class which itself represented, for the time
being, society as a whole:
%
\begin{itemize}
  \item{in ancient times, the State of slaveowning citizens;}
  \item{in the Middle Ages, the feudal lords;}
  \item{in our own times, the bourgeoisie.}
\end{itemize}
%
When at last, it becomes the real representative of the whole of society, it
renders itself unnecessary. As soon as there is no longer any social class to be
held in subjection; as soon as class rule, and the individual struggle for
existence based upon our present anarchy in production, with the collisions and
excesses arising from these, are removed, nothing more remains to be repressed,
and a special repressive force, a State, is no longer necessary. The first act
by virtue of which the State really constitutes itself the representative of the
whole of society---the taking possession of the means of production in the name
of society---that is, at the same time, its last independent act as a State.
State interference in social relations becomes, in one domain after another,
superfluous, and then dies out of itself; the government of persons is replaced
by the administration of things, and by the conduct of processes of production.
The State is not ``abolished''. It \emph{dies out}. This gives the measure of
the value of the phrase: ``a free State'', both as to its justifiable use at
times by agitators, and so it its ultimate scientific insufficiency; and also of
the demands of the so-called anarchists for the abolition of the State out of
hand.

Since the historical appearance of the capitalist mode of production, the
appropriation by society of all the means of production has often been dreamed
of, more or less vaguely, by individuals, as well as by sects, as the ideal of
the future. But it could become possible, could become a historical necessity,
only when the actual conditions for its realization were there. Like every other
social advance, it becomes practicable, not by men understanding that the
existence of classes is in contradiction to justice, equality, etc., not by the
mere willingness to abolish these classes, but by virtue of certain new economic
conditions. The separation of society into an exploiting and an exploited class,
a ruling and an oppressed class, was the necessary consequences of the deficient
and restricted development of production in former times. So long as the total
social labour only yields a produce which but slightly exceeds that barely
necessary for the existence of all; so long, therefore, as labour engages all or
almost all the time of the great majority of the members of society---so long,
of necessity, this society is divided into classes. Side by side with the great
majority, exclusively bond slaves to labour, arises a class freed from directly
productive labour, which looks after the general affairs of society: the
direction of labour, State business, law, science, art, etc. It is, therefore,
the law of division of labour that lies at the basis of the division into
classes. But this does not prevent this division into classes from being
carried out by means of violence and robbery, trickery and fraud. it does not
prevent the ruling class, once having the upper hand, from consolidating its
power at the expense of the working-class, from turning its social leadership
into an intensified exploitation of the masses. 

But if, upon this showing, division into classes has a certain historical
justification, it has this only for a given period, only under given social
conditions. It was based upon the insufficiency of production. It will be swept
away by the complete development of modern productive forces. And, in fact, the
abolition of classes in society presupposes a degree of historical evolution at
which the existence, not simply of this or that particular ruling class, but of
any ruling class at all, and, therefore, the existence of class distinction
itself, has become a obsolete anachronism. It presupposes, therefore, the
development of production carried out to a degree at which appropriation of the
means of production and of the products, and, with this, of political
domination, of the monopoly of culture, and of intellectual leadership by a
particular class of society, has become not only superfluous but economically,
politically, intellectually, a hindrance to development. 

This point is now reached. Their political and intellectual bankruptcy is
scarcely any longer a secret to the bourgeoisie themselves. Their economic
bankruptcy recurs regularly every 10 years. In every crisis, society is
suffocated beneath the weight of its own productive forces and products, which
it cannot use, and stands helpless, face-to-face with the absurd contradiction
that the producers have nothing to consume, because consumers are wanting. The
expansive force of the means of production burst the bonds that the capitalist
mode of production had imposed upon them. Their deliverance from these bonds is
the one precondition for an unbroken, constantly-accelerated development of the
productive forces, and therewith for a practically unlimited increase of
production itself. Nor is this all. The socialized appropriation of the means of
production does away, not only with the present artificial restrictions upon
production, but also with the positive waste and devastation of productive
forces and products that are at the present time the inevitable concomitants of
production, and that reach their height in the crises. Further, it sets free for
the community at large a mass of means of production and of products, by doing
away with the senseless extravagance of the ruling classes of today, and their
political representatives. The possibility of securing for every member of
society, by means of socialized production, an existence not only fully
sufficient materially, and becoming day-by-day more full, but an existence
guaranteeing to all the free development and exercise of their physical and
mental faculties---this possibility is now, for the first time, here, but
\emph{it is here}\endnote{
  A few figures may serve to give an approximate idea of the enormous expansive
  force of the modern means of production, even under capitalist pressure.
  According to Mr. Giffen, the total wealth of Great Britain and Ireland
  amounted, in round numbers in
  \begin{itemize}
    \item{1814 to £2\,200\,000\,000,}
    \item{1865 to £6\,100\,000\,000,}
    \item{1875 to £8\,500\,000\,000.}
  \end{itemize}
  As an instance of the squandering of means of production and of products
  during a crisis, the total loss in the German iron industry alone, in the
  crisis of 1873-78, was given at the second German Industrial Congress (Berlin,
  February 21, 1878) as £22\,750\,000.
}.

With the seizing of the means of production by society, production of
commodities is done away with, and, simultaneously, the mastery of the product
over the producer. Anarchy in social production is replaced by systematic,
definite organization. The struggle for individual existence disappears. Then,
for the first time, man, in a certain sense, is finally marked off from the rest
of the animal kingdom, and emerges from mere animal conditions of existence into
really human ones. The whole sphere of the conditions of life which environ man,
and which have hitherto ruled man, now comes under the dominion and control of
man, who for the first time become the real, conscious lord of nature, because
he has now become master of his own social organization. The laws of his own
social action, hitherto standing face-to-face with man as laws of Nature foreign
to, and dominating him, will then be used with full understanding, and so
mastered by him. Man's own social organization, hitherto confronting him as a
necessity imposed by Nature and history, now become the result of his own free
action. The extraneous objective forces that have, hitherto, governed history,
pass under the control of man himself. Only from that time will man himself,
more and more consciously, make his own history---only from that time will the
social causes set in movement by him have, in the main and in a constantly
growing measure, the results intended by him. It is the ascent of man from the
kingdom of necessity to the kingdom of freedom.

Let us briefly sum up our sketch of historical evolution.

\begin{enumerate}[label=\textbf{\arabic*.}]
  \item{
    \textbf{Medieval Society}---individual production on a small scale. Means of
    production adapted for individual use; hence primitive, ungainly, petty,
    dwarfed in action. Production for immediate consumption, either of the
    producer himself or his feudal lord. Only where an excess of production over
    this consumption occurs is such excess offered for sale, enters into
    exchange. Production of commodities, therefore, only in its infancy. But
    already it contains within itself, in embryo, anarchy in the production of
    society at large.
  }
  \item{
    \textbf{Capitalist Revolution}---transformation of industry, at first by
    means of simple cooperation and manufacture. Concentration of the means of
    production, hitherto scatted, into great workshops. As a consequence, their
    transformation from individual to social means of production---a
    transformation which does not, on the whole, affect the form of exchange.
    The old forms of appropriation remain in force. The capitalist appears. In
    his capacity as owner of the means of production, he also appropriates the
    products and turns them into commodities. Production has become a
    \emph{social} act. Exchange and appropriation continue to be
    \emph{individual} acts, the acts of individuals. The social product is
    appropriated by the individual capitalist. Fundamental contradiction, whence
    arise all the contradictions in which our present-day society moves, and
    which modern industry brings to light.
    \begin{enumerate}
      \item{
        Severance of the producer from the means of production. Condemnation of
        the worker to wage-labour for life. \emph{Antagonism between the
        proletariat and bourgeoisie.}
       }
       \item{
         Growing predominance and increasing effectiveness of the laws governing
         the production of commodities. Unbridled competition.
         \emph{Contradiction between socialized organization in the individual
         factory and social anarchy in the production as a whole.}
       }
       \item{
         On the one hand, perfecting of machinery, made by competition
         compulsory for each individual manufacturer, and complemented by a
         constantly growing displacement of labourers. \emph{Industrial
         reserve-army.} On the other hand, unlimited extension of production,
         also compulsory under competition, for every manufacturer. On both
         sides, unheard-of development of productive forces, excess of supply
         over demand, over-production and products---excess there, of labourers,
         without employment and without means of existence. But these two levers
         of production and of social well-being are unable to work together,
         because the capitalist form of production prevents the productive
         forces from working and the products from circulating, unless they are
         first turned into capital---which their very superabundance prevents.
         The contradiction has grown into an absurdity. \emph{The mode of
         production rises in rebellion against the form of exchange.}
       }
       \item{
         Partial recognition of the social character of the productive forces
         forced upon the capitalists themselves. Taking over of the great
         institutions for production and communication, first by joint-stock
         companies, later in by trusts, they by the State. The bourgeoisie
         demonstrated to be a superfluous class. All its social functions are
         now performed by salaried employees.
       }
    \end{enumerate}
  }
  \item{
    \textbf{Proletarian Revolution}---solution of the contradictions. The
    proletariat seizes the public power, and by means of this transforms the
    socialized means of production, slipping from the hands of the bourgeoisie,
    into public property. By this act, the proletariat frees the means of
    production from the character of capital they have thus far borne, and gives
    their socialized character complete freedom to work itself out. Socialized
    production upon a predetermined plan becomes henceforth possible. The
    development of production makes the existence of different classes of
    society thenceforth an anachronism. In proportion as anarchy in social
    production vanishes, the political authority of the State dies out. Man, at
    last the master of his own form of social organization, becomes at the same
    time the lord over Nature, his own master---free.
  }
\end{enumerate}

To accomplish this act of universal emancipation is the historical mission of
the modern proletariat. To thoroughly comprehend the historical conditions and
thus the very nature of this act, to impart to the now oppressed proletarian
class a full knowledge of the conditions and of the meaning of the momentous act
it is called upon to accomplish, this is the task of the theoretical expression
of the proletarian movement, scientific Socialism.

\printendnotes
