\documentclass[oneside, 12pt]{book}
\usepackage[utf8]{inputenc}

\usepackage{hyperref}

\title{Socialism: Utopian and Scientific}
\author{Friedrich Engels}
\date{1880}

\begin{document}

\maketitle

\tableofcontents

\part{Utopian Socialism}

\chapter{The Development of Utopian Socialism}

Modern Socialism is, in its essence, the direct product of the recognition, on
the one hand, of the class antagonisms existing in the society of today between
proprietors and non-proprietors, between capitalists and wage-workers; on the
other hand, of the anarchy existing in production. But, in its theoretical form,
modern Socialism originally appears ostensibly as a more logical extension of
the principles laid down by the great French philosophers of the 18th century.
Like every new theory, modern Socialism had, at first, to connect itself with
the intellectual stock-in-trade ready to its hand, however deeply its roots lay
in material economic facts.

The great men, who in France prepared men's minds for the coming revolution,
were themselves extreme revolutionists. They recognized no external authority of
any kind whatever. Religion, natural science, society, political
institutions---everything was subjected to the most unsparing criticism:
everything must justify its existence before the judgement-seat of reason or
give up existence. Reason became the sole measure of everything. It was the time
when, as Hegel says, the world stood upon its head [note]; first in the sense
that the human head, and the principles arrived at by its thought, claimed to be
the basis of all human action and association; but by and by, also, in the wider
sense that the reality which was in contradiction to these principles had, in
fact, to be turned upside down. Every form of society and government then
existing, every old traditional notion, was flung into the lumber-room as
irrational; the world had hitherto allowed itself to be led solely by
prejudices; everything in the past deserved only pity and contempt. Now, for the
first time, appeared the light of day, the kingdom of reason; henceforth
superstition, injustice, privilege, oppression, were to be superseded by
eternal truth, eternal Right, equality based on Nature and the inalienable
rights of man.


\chapter{Dialectics}

In the meantime, along with and after the French philosophy of the 18th century,
had arisen the new German philosophy, culminating in Hegel.

\end{document}
