\documentclass[oneside, 12pt]{book}
\usepackage[utf8]{inputenc}

\usepackage{enotez}
\usepackage{hyperref}

\title{Socialism: Utopian and Scientific}
\author{Friedrich Engels}
\date{1880}

\begin{document}

\maketitle

\tableofcontents

\chapter{The Development of Utopian Socialism}

Modern Socialism is, in its essence, the direct product of the recognition, on
the one hand, of the class antagonisms existing in the society of today between
proprietors and non-proprietors, between capitalists and wage-workers; on the
other hand, of the anarchy existing in production. But, in its theoretical form,
modern Socialism originally appears ostensibly as a more logical extension of
the principles laid down by the great French philosophers of the 18th century.
Like every new theory, modern Socialism had, at first, to connect itself with
the intellectual stock-in-trade ready to its hand, however deeply its roots lay
in material economic facts.

The great men, who in France prepared men's minds for the coming revolution,
were themselves extreme revolutionists. They recognized no external authority of
any kind whatever. Religion, natural science, society, political
institutions---everything was subjected to the most unsparing criticism:
everything must justify its existence before the judgement-seat of reason or
give up existence. Reason became the sole measure of everything. It was the time
when, as Hegel says, the world stood upon its head%
\endnote{
  This is the passage on the French revolution:
  \quote{
    ``Thought, the concept of law, all at once made itself felt, and against
    this the old scaffolding of wrong could make no stand. In this conception
    of law, therefore, a constitution has now been established, and henceforth
    everything must be based upon this. Since the Sun had been in the firmament,
    and the planets circled around him, the sight had never been seen of man
    standing upon his head---i.e., on the Idea---and building reality after this
    image. Anaxagoras first said that the \emph{nous}, Reason, rules the world;
    but now, for the first time, had men come to recognize that the Idea must
    rule the mental reality. And this was a magnificent sunrise. All thinking
    Beings have participated in celebrating this holy day. A sublime emotion
    swayed men at that time, an enthusiasm of reason pervaded the world, as if
    now had come the reconciliation of the Divine Principle with the world.''
    [Hegel's \emph{The Philosophy of History}, 1840, p. 535].
  }
}; first in the sense that the human head, and the principles arrived at by its
thought, claimed to be the basis of all human action and association; but by
and by, also, in the wider sense that the reality which was in contradiction to
these principles had, in fact, to be turned upside down. Every form of society
and government then existing, every old traditional notion, was flung into the
lumber-room as irrational; the world had hitherto allowed itself to be led
solely by prejudices; everything in the past deserved only pity and contempt.
Now, for the first time, appeared the light of day, the kingdom of reason;
henceforth superstition, injustice, privilege, oppression, were to be superseded
by eternal truth, eternal Right, equality based on Nature and the inalienable
rights of man.

We know today that this kingdom of reason was nothing more than the idealized
kingdom of the bourgeoisie; that this eternal Right found its realization in
bourgeois justice; that this equality reduced itself to bourgeois equality
before the law; that bourgeois property was proclaimed as one of the essential
rights of man; and that the government of reason, the \emph{Contract Social} of
Rousseau, came into being, and only could come into being, as a democratic
bourgeois republic. The great thinkers of the 18th century could, no more than
their predecessors, go beyond the limits imposed upon them by their epoch.

But, side by side with the antagonisms of the feudal nobility and the burghers,
who claimed to represent all the rest of society, was the general antagonism of
exploiters and exploited, of rich idlers and poor workers. It was this very
circumstance that made it possible for the representatives of the bourgeoisie to
put themselves forward as representing not one special class, but the whole of
suffering humanity. Still further. From its origin the bourgeoisie was saddled
with its antithesis: capitalists cannot exist without wage-workers, and, in the
same proportion as the mediaeval burgher of the guild developed into the modern
bourgeois, the guild journeyman and the day-laborer, outside the guilds,
developed into the proletarian. And although, upon the whole, the bourgeoisie,
in their struggle with the nobility, could claim to represent at the same time
the interests of the different working-classes of that period, yet in every
great bourgeois movement there were independent outbursts of that class which
was the forerunner, more or less developed, of the modern proletariat. For
example, at the time of the German Reformation and the Peasants' War, the
Anabaptists and Thomas Müntzer; in the great English Revolution, the Levellers;
in the great French Revolution, Babeuf.

These were theoretical enunciations, corresponding with these revolutionary
uprisings of a class not yet developed; in the 16th and 17th centuries, Utopian
pictures of ideal social conditions; in the 18th century, actual communistic
theories (Morelly and Mably). The demand for equality was no longer limited to % TODO add note
political rights; it was extended also to the social conditions of individuals.
It was not simply class privileges that were to be abolished, but class
distinctions themselves. A Communism, ascetic, denouncing all the pleasures of
life, Spartan, was the first form of the new teaching. Then came the three great
Utopians: Saint-Simon, to whom the middle-class movement, side by side with the
proletarian, still had a certain significance; Fourier; and Owen, who in the
country where capitalist production was most developed, and under the influence
of the antagonisms begotten of this, worked out his proposals for the removal of
class distinction systematically and in direct relation to French materialism.

One thing is common to all three. Not one of them appears as a representative of
the interests of that proletariat which historical development had, in the
meantime, produced. Like the French philosophers, they do not claim to
emancipate a particular class to begin with, but all of humanity at once. Like
them, they wish to bring in the kingdom of reason and eternal justice, but this
kingdom, as they see it, is as far as Heaven from Earth, from that of the French
philosophers.

For, to our three social reformers, the bourgeois world, based upon the
principles of these philosophers, is quite as irrational and unjust, and,
therefore, finds its way to the dust-hole quite as readily as feudalism and all
the earlier stages of society. If pure reason and justice have not, hitherto,
ruled the world, this has been the case only because men have not rightly
understood them. What was wanted was the individual man of genius, who has now
arisen and who understands the truth. That he has now arisen, that the truth has
now been clearly understood, is not an inevitable event, following of necessity
in the chains of historical development, but a mere happy accident. He might
just as well have been born 500 years earlier, and might then have spared
humanity 500 years of error, strife, and suffering.

We saw how the French philosophers of the 18th century, the forerunners of the
Revolution, appealed to reason as the sole judge of all that is. A rational
government, rational society, were to be founded; everything that ran counter to
eternal reason was to be remorselessly done away with. We saw also that this
eternal reason was in reality nothing but the idealized understanding of the
18th century citizen, just then evolving into the bourgeois. The French
Revolution had realized this rational society and government. 

\printendnotes


\chapter{Dialectics}

In the meantime, along with and after the French philosophy of the 18th century,
had arisen the new German philosophy, culminating in Hegel.

\end{document}
